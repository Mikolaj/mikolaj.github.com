\section{\secactual}
\label{sec:actual}

In this section I'll try to describe how I actually implemented the algorithm
sketched in the Theorem, on the basis of the ML Kit algorithm for
the SML signature elaboration.

\subsection{\secconventions}
\label{sec:conventions}

\subsection{\secimplaxioms}
\label{sec:impl_axioms}

First I would like to obtain $(\N')\S'$ and the realisation. 
My implementation elaborates the original $\sigexp$ 
instead of its stripped version taking advantage of the fact, 
that errors found while elaborating axioms are recorded in the
abstract syntax tree and in other ways ignored.
The result of this elaboration is some strange trace $\trace''$, 
the principal signature $(\N'')\S''$ for $\sigexp'$,
and the corresponding realisation. Now I make the signature equality-principal,
and accordingly extend the realisation, getting at last $(\N')\S'$. This is the end of the first pass.
The code follows:
{\small
\begin{verbatim}
fun elab_psigexp (B: Env.Basis, psigexp: IG.psigexp)
    : (Stat.Sig * OG.psigexp) =
  let
    val IG.PRINCIPpsigexp(i, sigexp) = psigexp
    val A = Env.mkAssembly B
(* mikon#1.45 *)
    val NofB = Env.N_of_B B
    val (rea, _, S_first, _) = 
        (StrId.backup_state();
        elab_sigexp'(B, sigexp, A, FIRST_PASS))
    val (_, rea1) = Stat.equality_principal(NofB, S_first)
    val (_, _, S, out_sigexp) = 
        (StrId.restore_state();
        elab_sigexp'(B, sigexp, A, SECOND_PASS(rea1 oo rea)))
(* end mikon#1.45 *)
  in ... S ... out_sigexp ... end
\end{verbatim}
}

Now the only things that I need are to check that axioms are valid, 
get their traces, and repair $\trace''$ to get the required trace.
The easiest way achieve these goals is to elaborate the $\sigexp$ for the second
time, with an additional parameter being the realisation.
When axioms are found they are elaborated in the bases
modified by the realisation:
{\small
\begin{verbatim}
fun elab_spec (B: Env.Basis, spec: IG.spec, A: Env.Assembly, p : pass): 
   (Stat.Realisation * Env.Assembly * Env.Env * OG.spec) =
  case spec of
  ...
(* mikon#1.4 and #1.43 *)
   (* Axiom specification *)
  | IG.AXIOMspec(i, axdesc) =>
      let
        val (out_axdesc) = 
          case p
            of FIRST_PASS => elab_axdesc(B, axdesc)
             | (SECOND_PASS(rea1)) => elab_axdesc(rea1 onB B, axdesc)
      in
        (Stat.Id, Env.emptyA, Env.emptyE, OG.AXIOMspec(okConv i, out_axdesc))
      end
(* end mikon#1.4 and #1.43 *)
\end{verbatim}
}

\subsection{\secimpltraces}
\label{sec:impl_traces}
 
%Another possibility is to refrain from correcting the trace during the elaboration,
%but instead apply the final substitution to the trace obtained at the end of the process.
%To prove that it works a lemma is needed, similar to the one proved in Section~\ref{sec:comforting}.
%The proof of the lemma proceeds by induction over all the rules of the Static Semantics of EML.

\section{\secanegdote}
\label{sec:anegdote}

When I had finished coding this algorithm, for a few days I was sure, I could prove,
that the algorithm was correct, but the algorithm kept producing strange
and inexplicable results. I was a bit annoyed. Then what made me really angry
was the discovering that the culprits were side-effects used in some places
in the ML Kit for example to get fresh type names.

I solved the problem in a rather brutal way, as can be seen
in the code for the function \|elab_psigexp| above
and in the offending module below:
{\small
\begin{verbatim}
functor Timestamp(): TIMESTAMP =
struct
  type stamp = int
  val r = ref 0
  fun new() = (r := !r + 1; !r)
  fun print i = "$" ^ Int.string i
(* mikon#1.44 *)
  val backup = ref 0
  fun backup_state() = backup := !r
  fun restore_state() = r := !backup
(* end mikon#1.44 *)
end; 
\end{verbatim} 
} %$








%%% Local Variables: 
%%% mode: latex
%%% TeX-master: "eml-type"
%%% End: 










