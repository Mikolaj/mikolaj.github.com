    % My title
\newcommand{\MyTitle}
    {%\bf  
\mbox{Analiza~statyczna~w~Extended~ML}
%\\             szkic
           }

    % My author
\newcommand{\MyAuthor}
    {Miko"laj Konarski
             %\\
             %\thanks{Institute of Informatics, Warsaw University, Warsaw, Poland.}\\
             %{\tt mikon@mimuw.edu.pl}
        }

\newcommand{\MyDate}
    {22 sierpnia 1997}

    % Amsthm names 
\newcommand{\Theorem}{Twierdzenie}
\newcommand{\Corollary}{Wniosek}
\newcommand{\Lemma}{Lemat}
\newcommand{\Observation}{Spostrze"renie}
\newcommand{\Definition}{Definicja}

    % Theorem names 
\newcommand{\thmcomforting}{Pocieszaj"acy Lemat}

    % Part, section, etc. names
\newcommand{\partpreface}{Przedmowa}
\newcommand{\partintro}{Wst"ep}
\newcommand{\parttheory}{Teoria}
\newcommand{\partimpl}{Implementacja}
\newcommand{\partget}{Dodatki}

\newcommand{\secover}{Rzut oka na dziedzin"e}
\newcommand{\secsml}{Standard ML}
\newcommand{\seceml}{Extended ML}
\newcommand{\seckit}{EML Kit}
\newcommand{\secformal}{Semantyka naturalna}
\newcommand{\secdepend}{Podsumowanie i diagram zale"rno"sci}

\newcommand{\secanal}{Analiza statyczna}
\newcommand{\secall}{Obja"snienie terminu}
\newcommand{\secbasic}{Podstawowe poj"ecia}
\newcommand{\secsemantic}{Obiekt semantyczny}
\newcommand{\secjudgments}{Os"ad}
\newcommand{\secprincipal}{Obiekt g"l"owny}
\newcommand{\secinterest}{Ciekawsze zjawiska}
\newcommand{\secindeterm}{Niedeterminizm}
\newcommand{\secimposing}{Wymuszanie}
\newcommand{\secanaleml}{Analiza statyczna w Extended ML}

\newcommand{\secaxiom}{Jak bada"c poprawno"s"c aks\-jo\-ma\-t"ow}

\newcommand{\secsketch}{Szkic algorytmu}
\newcommand{\secinsig}{Sygnatury z aksjomatami}
\newcommand{\secquasi}{Analiza statyczna modu"l"ow w QUASI-EML}
\newcommand{\secnaive}{Naiwna pr"oba rozwi"azania problemu}
\newcommand{\seccomforting}{Pocieszaj"acy lemat}
\newcommand{\secmature}{Poprawne rozwi"azanie}
\newcommand{\secvalidating}{Badanie dopuszczalno"sci aksjomat"ow}

\newcommand{\secinter}{Interludium}
\newcommand{\secstatic}{Semantyka statyczna sygnatur}
\newcommand{\secelaborating}{Elaboracja sygnatur}

\newcommand{\secsearch}{W poszukiwaniu bazy $\B_e$}
\newcommand{\secmodest}{Baza pocz"atkowa}
\newcommand{\secignorant}{Baza bie"r"aca}
\newcommand{\secomniscient}{Baza ko"ncowa}
\newcommand{\secobservation}{Spostrze"renie}

\newcommand{\sectrace}{Jak gromadzi"c "slady}

\newcommand{\secpreludium}{Preludium}
\newcommand{\secstatexp}{Semantyka statyczna wyra"re"n}
\newcommand{\secelabexp}{Elaboracja wyra"re"n}

\newcommand{\sectraces}{"Slady}
\newcommand{\sectraceclos}{"Slad elaboracji} 
\newcommand{\sectracedef}{"Slad jako obiekt semantyczny}

\newcommand{\secgather}{Zbieranie "slad"ow}
\newcommand{\secsubstitutions}{Lokalna i globalna g"l"owno"s"c}
\newcommand{\secscheme}{Konkretyzacja "sladu}

\newcommand{\secimplaxioms}{Aksjomaty w sygnaturach}
\newcommand{\secimpltracesmod}{"Slady j"ezyka Modu"l"ow}
\newcommand{\secimpltraces}{"Slady j"ezyka J"adra}
\newcommand{\secanegdote}{Anegdota}

    % misc
\newcommand{\statsem}{\mbox{$\cal S$}}





%%% Local Variables: 
%%% mode: latex
%%% TeX-master: "eml-type"
%%% End: 
