
\section{System EML Kit}
\label{sec:system}

\paragraph{EML Kit jest dost"epny w sieci Internet.}
Poza wydaniem kodu "zr"o\-d"lo\-we\-go systemu EML Kit wersji 1.0,
istnieje r"ownie"r wydanie zawieraj"ace 
binaria dla systemu Solaris na architekturze Sun SPARC,
oraz wydanie zawieraj"ace binaria dla systemu Linux
na architekture i386.

Wszystkie te zestawy s"a dost"epne przez anonimowe ftp z: 
\begin{verbatim}
ftp://zls.mimuw.edu.pl/pub/mikon/EMLKit
\end{verbatim}

Mo"rna je r"ownie"r znale"z"c na stronie domowej WWW systemu EML Kit: 
\begin{verbatim}
http://zls.mimuw.edu.pl/~mikon/ftp/EMLKit/README.html
\end{verbatim}

\paragraph{EML Kit ma nast"epuj"acy copyright:}
\begin{center}
Copyright (C) 1993 Edinburgh and Copenhagen Universities:\\ for the ML~Kit\\
Copyright (C) 1996, 1997 Marcin Jurdzi"nski, Miko"laj Konarski,\\ S"lawomir Leszczy"nski i Aleksy Schubert
\end{center}

EML Kit zosta"l wydany na warunkach opisanych w licencji GNU GPL, co wyja"sniaj"a i przypiecz"etowuj"a
nas\-t"e\-pu\-j"a\-ce informacje w j"ezyku angielskim za"l"aczane do ka"rdego wydania systemu:
\begin{quotation}  
The EML Kit is free software; you can redistribute it and/or modify
it under the terms of the GNU General Public License as published by
the Free Software Foundation; either version 2 of the License, or
(at your option) any later version.

The EML Kit is distributed in the hope that it will be useful,
but WITHOUT ANY WARRANTY; without even the implied warranty of
MERCHANTABILITY or FITNESS FOR A PARTICULAR PURPOSE.  See the
GNU General Public License for more details.
\end{quotation}

\pagebreak

\paragraph{Autorzy systemu EML Kit s"a wdzi"eczni za wszelk"a pomoc.}
Prace nad systemem EML Kit by"ly cz"e"sciowo wspierane przez
nast"epuj"ace granty KBN (Komitetu Bada"n Naukowych):
\begin{itemize}
\item "`Formal Methods of Software Development''
\begin{verbatim}
http://wwwat.mimuw.edu.pl/~tarlecki/kbn93grant/index.html
\end{verbatim}
\item "`Logical Aspects of Software Specification and Development Methods''
\begin{verbatim}
http://wwwat.mimuw.edu.pl/~tarlecki/kbn96grant/index.html
\end{verbatim}
\end{itemize}

\section{Obiekty semantyczne}
\label{sec:semobjeml}

Dziedziny prostych obiekt"ow semantycznych Semantyki Statycznej J"adra Ex\-tend\-ed~ML,
przestawione s"a w Tabeli~\ref{simple-objects}.
Dziedziny te: $\TyVar$, $\TyNames$ i $\StrNames$, to dowolne niesko"nczone zbiory.
Ponadto ka"rdy $\t\in\TyNames$ posiada arno"s"c $k\geq 0$, 
a klasa nazw typ"ow o arno"sci $k$ znaczana jest $\TyNamesk$.

%\vspace{-7mm}
\begin{figure}[h]
%\vspace{4pt}
\begin{displaymath}
\begin{array}{rclr}
\alpha\index{\alpha@{$\alpha$ (zmienne typowe)}}\ {\rm lub}\ \tyvar
& \in   & \TyVar        & \mbox{zmienne typowe}\\
\t\index{t (type name)}               & \in   & \TyNames\index{TyName}  & \mbox{nazwy typ"ow}\\
\m\index{m (structure name)}            & \in   & \StrNames\index{StrName}      & \mbox{nazwy struktur}\\
\end{array}
\end{displaymath}
\caption{Proste obiekty semantyczne}
\label{simple-objects}
%\vspace{6pt}
\end{figure}

Dziedziny z"lo"ronych obiekt"ow semantycznych Semantyki Statycznej J"a\-dra EML,
przestawione s"a w Tabeli~\ref{compound-objects}.

Gdy $A$ i $B$ s"a zbiorami, $\Fin A$ oznacza zbi"or sko"nczonych podzbior"ow $A$,
za"s $\finfun{A}{B}$ oznacza zbi"or sko"nczonych odwzorowa"n
(cz"e"sciowych funkcji o sko"nczonej dziedzinie) z $A$ do $B$.

Dla dowolnej dziedziny obiekt"ow semantycznych $A$ definiujemy
$\Tree{A}$ jako zbi"or sko"nczonych binarnych drzew o elementach z $A$.
Innymi s"lowy $\Tree{A}$ jest najmniejszym rozwi"azaniem r"ownania dziedzinowego
$$
\Tree{A} = \{\emptylist\}\uplus
A\uplus\{\append{x}{y}~|~x,y\in \Tree{A}\}
$$

Operacja $\uplus$ oznacza sum"e roz"l"aczn"a dziedzin.
Zak"ladamy, "re wszystkie definiowane dziedziny s"a roz"l"aczne.

%\pagebreak

\begin{figure}
%\vspace{2pt}
\begin{displaymath}
\begin{array}{rcl}
        \tau\index{\tau@{$\tau$ (type)}}        
&\in    &\Type\index{Type} = \TyVar\uplus\RecType\uplus\FunType
                                 \uplus\ConsType\\
 \longtauk\index{\tau k@{$\tauk$ (type vector)}}\ {\rm lub}\ \tauk
                & \in   & \Type^k\\
 \longalphak\ {\rm lub}\ \alphak
                & \in   & \TyVar^k\\
 \varrho\index{\varrho@{$\varrho$ (record type)}}
                & \in   & \RecType\index{RecType} = \finfun{\Lab}{\Type} \\
 \tau\rightarrow\tau'\index{\to@{$\tau\to\tau'$ (function type)}}
                & \in   & \FunType\index{FunType} = \Type\times\Type \\
                &       & \ConsType\index{ConsType} = \uplus_{k\geq 0}\ConsType^{(k)}\\
        \tauk\t & \in   & \ConsType^{(k)} = \Type^k\times\TyNamesk  \\
\typefcn\ {\rm lub}\ \typefcnk\index{\theta@{$\typefcn$ (type function)}}%
\index{\Lambda@{$\typefcnk$ (type function)}}
                & \in   & \TypeFcn\index{TypeFcn} = \uplus_{k\geq 0}\TyVar^k\times\Type\\
\tych\ {\rm lub}\ \longtych
                & \in   & \TypeScheme = \uplus_{k\geq 0}\TyVar^k\times\Type\\
\S\ {\rm lub}\ (\m,\E)
                & \in   & \Str = \StrNames\times\Env \\
(\theta,\CE)    & \in   & \TyStr = \TypeFcn\times\ConEnv\\
\SE             & \in   & \StrEnv = \finfun{\StrId}{\Str}\\
\TE             & \in   & \TyEnv = \finfun{\TyCon}{\TyStr}\\
\CE             & \in   & \ConEnv = \finfun{\Id}{\TypeScheme}\\
\VE             & \in   & \VarEnv =
\finfun{\Id}{\TypeScheme}\\
\E\ {\rm lub}\ \longE{}
                & \in   & \Env = \StrEnv\times\TyEnv\times\VarEnv\\
\T              & \in   & \TyNameSets = \Fin(\TyNames)\\
\U              & \in   & \TyVarSet = \Fin(\TyVar)\\
\C\ {\rm lub}\ \T,\U,\E   & \in   & \Context\index{context!static semantics} =
\TyNameSets\times\TyVarSet\times\Env\\
\tyrea & \in & \TyRea = \TyNames\to\TypeFcn\\
\trace & \in & \Trace = \Tree{\mbox{\SimTrace} \uplus\TraceScheme}\\
&& \SimTrace = \Type\uplus\Env\uplus(\Context\times\Type)\uplus\phantom{\Env}\\
&& \qquad (\Context\times\Env)\uplus(\Context\times\TyNames)\uplus\phantom{\Env}\\
&& \qquad \TyEnv\uplus(\VarEnv\times\TyRea)\\
&& \TraceScheme = \uplus_{k\geq 0} \TraceScheme^{(k)}\\
\forall\alphak.\trace & \in &
\TraceScheme^{(k)}=\mbox{TyVar}^k\times\Trace
\end{array}
\end{displaymath}
\caption{Z"lo"rone obiekty semantyczne}
\label{compound-objects}
%\vspace{3pt}
\end{figure}

\clearpage

Obiekty semantyczne Semantyki Statycznej Modu"l"ow EML
to obiekty semantyczne Semantyki Statycznej J"adra EML,
oraz dodatkowe obiekty semantyczne przestawione w Tabeli~\ref{module-objects}.

\begin{figure}[h]
%\vspace{2pt}
\begin{displaymath}
\begin{array}{rcl}
\M              & \in   & \StrNameSets = \Fin(\StrNames)\\
\N\ {\rm lub}\ (\M,\T)
                & \in   & \NameSets = \StrNameSets\times\TyNameSets\\
\sig\ {\rm lub}\ \longsig{}
                & \in   & \Sig =  \NameSets\times\Str \\
\funsig\ {\rm lub}\ \longfunsig{}
                & \in   & \FunSig = \NameSets\times
                                         (\Str\times\Sig)\\
\G              & \in   & \SigEnv        =       \finfun{\SigId}{\Sig} \\
\F              & \in   & \FunEnv        =       \finfun{\FunId}{\FunSig} \\
\B\ {\rm lub}\ \N,\F,\G,\E
                & \in   & \Basis = \NameSets\times
                                              \FunEnv\times\SigEnv\times\Env\\
\strrea         & \in   & \StrRea=\StrNames\to\StrNames\\
\rea\ {\rm lub}\ \longrea{}
                & \in   & \Rea=\TyRea\times\StrRea\\
\trace &\in&\Trace=\Tree{\Trace_{\rm COR}\uplus\SimTrace\uplus\BoundTrace}\\
(\N)\trace&\in&\BoundTrace=\NameSets\times\Trace\\
&&\SimTrace = \StrNames\uplus\Rea\uplus\VarEnv\uplus\TyEnv\uplus\Env\\
\end{array}
\end{displaymath}
\caption{Dalsze z"lo"rone obiekty semantyczne}
\label{module-objects}
%\vspace{3pt}
\end{figure}

\clearpage

%\section{Przyk"ladowa elaboracja}
%\label{sec:examplelab}
%%%%%%%?????????????przyklad elaboracji prostego programu
%ze srodowiskami, sladai, etc.



%%% Local Variables: 
%%% mode: latex
%%% TeX-master: "eml-type"
%%% End: 
