\section{\sectrace}
\label{sec:trace}

Skomplikowanie operacji zbierania "slad"ow wynika z faktu, 
"re drzewa wywodu, odpowiadaj"ace poszczeg"olnym stadiom 
elaboracji wyra"renia, nie s"a poddrzewami ostatecznego, 
poprawnego drzewa wywodu tego wy\-ra\-"re\-nia.
Przed przy\-st"a\-pie\-niem do rozwa"ra"n na temat "slad"ow
spr"obuj"e wyja"sni"c to zjawisko.

\subsection{\secpreludium}
\label{sec:preludium}

Przedstawiane w tym rozdziale przyk"lady pochodz"a z cz"e"sci wsp"olnej EML i SML.
Elaboracj"a konstrukcji specyfikacyjnych J"adra EML rz"adz"a podobne do opisanych poni"rej zasady.

\subsubsection{\secstatexp}
\label{sec:statexp}

Podobnie jak w Semantyce Statycznej Modu"l"ow (por"ownaj rozdzia"l~\ref{sec:static}),
w Semantyce Statycznej J"adra du"r"a rol"e gra niedeterminizm.
W wielu w"ez"lach wywod"ow, typy przypisywane zmiennym s"a "`niedeterministycznie zgadywane'',
tak aby mog"ly by"c zastosowane regu"ly, zak"ladaj"ace r"owno"s"c nie\-kt"o\-rych typ"ow
pojawiaj"acych si"e w przes"lankach. Niedeterminizm wys\-t"e\-pu\-je na przyk"lad w regule~35 (uproszczonej):
$$
\frac{}
     {\C\ts\var\ra (\{\var\mapsto\tau\},\tau)}
\eqno(35)
$$
Formalnemu argumentowi funkcji mo"re tu zosta"c przypisany dowolny typ.
Za"s w regule~2 (nieco uproszczonej):
$$
\frac{\C(\longvar)\succ\tau}
     {\C\ts\longvar\ra\tau}
\eqno(2)
$$
typ zmiennej mo"re by"c dowolnie wybrany spo"sr"od instancji
typu tej zmiennej przechowywanego w "srodowisku.

\subsubsection{\secelabexp}
\label{sec:elabexp}

Lekarstwem na niemo"rliwy do bezpo"sredniego zaimplementowania niedeterminizm
jest algorytm Damasa i Milnera~\cite{DM82}.

Elaboruj"ac wyra"renie przy pomocy tego algorytmu,
wywodzi si"e typy g"l"owne podwyra"re"n,
konkretyzuj"ac w razie potrzeby konteksty, w kt"orych odbywa si"e elaboracja.
Zamiast "`zgadywa"c'' typy, zak"lada si"e "re s"a one 
najbardziej og"olne, czyli "re s"a "swie"rymi zmiennymi typowymi.
W wypadku regu"l, zak"ladaj"acych r"owno"s"c pewnych typ"ow w przes"lankach, 
przeprowadza si"e unifikacj"e.
Aplikacja podstawie"n wynik"lych z unifikacji reprezentuje
precyzowanie wiedzy na temat tego, jakie typy powinny by"ly by"c
"`zgadni"ete'' w poprzednich stadiach elaboracji.

W regule typowania aplikacji (nieco uproszczonej):
$$
\frac{\C\ts\exp\ra\tau'\rightarrow\tau,\trace\qquad\C\ts\atexp\ra\tau',\trace'}
     {\C\ts\appexp\ra\tau,\append{\trace}{\trace'}}
\eqno(10)
$$
wida"c wymaganie r"owno"sci typu $\tau'$ w obu przes"lankach.
A oto fragment systemu EML Kit odpowiadaj"acy tej regule.
\|S1|, \|S2| i \|S3| s"a tu podstawieniami, \|oo| ich z"lo"reniem,
a \|onC| i \|on| aplikacj"a podstawienia odpowiednio do kontekstu i typu:
{\small
\begin{verbatim}
fun elab_exp (C : Environments.Context, exp : IG.exp) : 
    (Substitution * StatObject.Type * OG.exp) =
    case exp of
    ...
    (* Application expression *)                      (* rule 10 *)
    | IG.APPexp(i, exp, atexp) => 
        let
            val (S1, tau1, out_exp)   = elab_exp(C, exp)
            val (S2, tau2, out_atexp) = elab_atexp(S1 onC C, atexp)
            val new   = freshType()
            val arrow = StatObject.mkTypeArrow(tau2,new) 
            val (S3, i') = Unify(arrow, S2 on tau1, i)
        in
            (S3 oo S2 oo S1, S3 on new, 
             OG.APPexp(i', out_exp, out_atexp))
        end
\end{verbatim}
}
Wida"c, jak zamiast "`niedeterministycznego zgadywania'' 
u"rywa si"e tu "swie"rego typu (\|freshType|).
Mo"rna zauwa"ry"c unifikacj"e (\|Unify|) i aplikacje podstawie"n.
Wyra"zna jest te"r rola umowy, "re ka"rda elaboracja podwyra"renia
daje w wyniku w szczeg"olno"sci wytworzone przez siebie podstawienie.

\subsection{\sectraces}
\label{sec:traces}

"Slady s"lu"r"a akumulowaniu obiekt"ow semantycznych pojawiaj"acych si"e podczas analizy statycznej. 
Do informacji przechowywanej w "sladach odwo"luje si"e nast"epnie Semantyka Weryfikacyjna EML.
"Slady graj"a r"ownie"r pewn"a aktywn"a rol"e w Semantyce Statycznej EML,
chocia"r post"epuj"ac z wystarczaj"ac"a ostro"rno"sci"a mo"rna,
jak pokaza"lem w rozdziale~\ref{sec:search}, poprawnie elaborowa"c programy bez ich u"rycia.

\subsubsection{\sectraceclos}
\label{sec:traceclos}

W rozdziale~\ref{sec:judgments} "slady zosta"ly przedstawione, na potrzeby 
rozwa"ra"n o analizie statycznej, jako drzewa wywodu os"ad"ow semantyki statycznej. 
Jest to o tyle usprawiedliwione, i"r ze "sladu mo"rna zrekonstruowa"c
reprezentowany przez niego wyw"od, 
cho"c jak wspomina"lem w rozdziale~\ref{sec:inter}, nie jest to proste.

Od tej pory, przez "slad analizy statycznej programu $P$ w kontek"scie $C$, b"ed"e
rozumia"l g"l"owny "slad, jaki wed"lug Semantyki Statycznej EML 
(patrz rozdzia"l~\ref{sec:tracedef}) daje si"e wywie"s"c dla $P$ w $C$.
Natomiast przez "slad elaboracji programu $P$ w kontek"scie $C$, b"ed"e
rozumia"l "slad analizy statycznej $P$ w~$\rea(C)$, gdzie $\rea$ jest
podstawieniem wynik"lym z elaboracji $P$ w $C$.

Na przyk"lad, podczas elaboracji programu:
\begin{verbatim}
fun f x = (x 5) < 25
\end{verbatim}
wyra"renie $P$, wyst"epuj"ace po prawej stronie znaku r"owno"sci, 
elaborowane jest w pewnym kontek"scie $C$. W tym kontek"scie zmiennej \|x| przypisany jest typ, 
b"ed"acy zmienn"a typow"a, nazwijmy j"a $\alpha$.
Wida"c, "re nie istnieje "slad analizy statycznej $P$ w kontek"scie $C$,
poniewa"r typ przypisany w $C$ zmiennej \|x| nie jest typem funkcyjnym,
a wi"ec nie da si"e wywie"s"c typu dla aplikacji \|(x 5)|.
Natomiast "slad elaboracji $P$ w kontek"scie $C$ istnieje,
gdy"r r"owny jest on "sladowi analizy statycznej $P$ w kontek"scie~$\rea(C)$,
gdzie podstawienie $\rea$, wynikaj"ace z elaboracji $P$, 
przypisuje zmiennej $\alpha$ typ $\INT\rightarrow\INT$.

\subsubsection{\sectracedef}
\label{sec:tracedef}

"Slady posiadaj"a pewn"a dodatkow"a struktur"e, wynikaj"ac"a z mo"rliwo"sci ich domykania.
Operator ${\rm Clos}$ jest u"rywany w kilku miejscach Semantyki Statycznej J"adra EML
(regu"ly 11.1, 11.2, 11.3 i 17), aby domyka"c "slad wzgl"edem kontekstu:
$$\cl{C}{\trace}=\forall\alphak.\trace$$
gdzie $\forall$ to kwantyfikator wi"a"r"acy zmienne typowe, 
za"s $\alphak$ to zmienne typowe wolne w $\trace$, a nie wyst"epuj"ace wolno w $C$.
W "sladach Semantyki Statycznej Modu"l"ow EML mo"rna natomiast wi"aza"c 
nazwy struktur lub nazwy typ"ow.

A oto pe"lna definicja "slad"ow J"adra EML:
\begin{displaymath}
\begin{array}{l}
\Trace = \Tree{\mbox{\SimTrace} \uplus\TraceScheme}\\
\SimTrace = \Type\uplus\Env\uplus(\Context\times\Type)\uplus(\Context\times\Env)\uplus\phantom{\Env}\\
\qquad (\Context\times\TyNames)\uplus\TyEnv\uplus(\VarEnv\times\TyRea)\\
\TraceScheme = \uplus_{k\geq 0} \TraceScheme^{(k)}\\
\TraceScheme^{(k)}=\mbox{TyVar}^k\times\Trace
\end{array}
\end{displaymath} 
gdzie $\Tree{A}$ jest dziedzin"a sko"nczonych binarnych drzew element"ow z $A$
(st"ad te"r pochodzi spotykany dalej operator sk"ladania "slad"ow,
pojawiaj"acy si"e we frazach w rodzaju $\append{\trace}{\trace'}$).
Pozosta"le nazwy oznaczaj"a albo dziedziny obiekt"ow Semantyki Statycznej EML,
albo zbiory "slad"ow szcze\-g"ol\-nej postaci, jak np.\ $\TraceScheme$,
kt"ory zawiera "slady b"ed"ace domkni"eciami innych "slad"ow.

Prezentacja "slad"ow Modu"l"ow EML w definicji Extended~ML~\cite{bib:KST94} 
nie jest w pe"lni satysfakcjonuj"aca,
g"l"ownie z punktu widzenia osoby zajmuj"acej si"e implementacj"a.
W nowej wersji definicji Extended ML~\cite{KST97} autorzy
% w swojej "ryczliwo"sci 
zdefiniowali "slady zgodnie z moj"a propozycj"a zawart"a w~\cite{JKS96}:
\begin{displaymath}
\begin{array}{l}
\Trace = \Tree{\Trace_{\rm COR}\uplus\SimTrace\uplus\BoundTrace}\\
\BoundTrace=\NameSets\times\Trace\\
\SimTrace = \StrNames\uplus\Rea\uplus\VarEnv\uplus\TyEnv\uplus\Env\\
\end{array}
\end{displaymath}
gdzie $\Trace_{\rm COR}$ to "slady J"adra EML, opisane powy"rej.
Zgodnie z t"a definicj"a przeprowadzi"lem implementacj"e zbierania "slad"ow
i ta definicja pos"lu"ry mi za podstaw"e rozwa"ra"n
o "sladach, metodzie uzyskiwania "slad"ow elaboracji program"ow i poprawno"sci tej metody.

\subsection{\secgather}
\label{sec:gather}

\subsubsection{\secsubstitutions}
\label{sec:substitutions}

Zbieranie "slad"ow prostych jest trywialne.
Nieco k"lopot"ow nastr"eczaj"a natomiast "slady,
kt"ore uzyskuje si"e przez domkni"ecie innych "slad"ow.
Przyk"ladem mo"re by"c regu"la 17 (tutaj nieco uproszczona):
$$
\frac{
\C\ts\valbind\ra\VE,\trace\qquad
\VE'=\cl{\C}{\VE}}
     {\C\ts\valdec\ra\VE'\ \In\ \Env,\cl{\C}{\trace}}
\eqno(17)
$$
gdzie "sladem jaki nale"ry wywie"s"c dla frazy "`$\VAL\ \valbind$'' jest
domkni"ecie "sladu $\trace$, wywiedzionego dla
jedynego podtermu tej frazy --- "`$\valbind$''.
Twierdzenia podobne do poni"rszego, cho"c nieco bardziej skomplikowane,
mo"rna dowie"s"c r"ownie"r dla innych regu"l, w kt"orych na\-st"e\-pu\-je domykanie "slad"ow.

\begin{thm}
\label{clos_traces}
Niech program $P$ ma posta"c $\VAL\ P'$.
Niech $C$ b"edzie kontekstem, $\trace$ "sladem elaboracji $P'$
w kontek"scie $C$, za"s $\rea$ niech b"edzie podstawieniem
wynik"lym z tej elaboracji.

Wtedy "sladem elaboracji $P$ w kontek"scie $C$ jest $\cl{\rea(C)}{\trace}$.
\end{thm}

\begin{proof}
Wynikiem elaboracji $P$ w kontek"scie $C$ jest wywiedzenie g"l"ownego obiektu semantycznego
dla $P$ w kontek"scie $\rea(C)$ wed"lug regu"ly 17, lecz pomijaj"ac "slady. 
Skoro tak, to mo"rna zastosowa"c regu"l"e 17
dla programu $P$ i kontekstu $\rea(C)$ i otrzyma"c tez"e.
\end{proof}

Jednak najciekawszy jest przypadek "slad"ow z"lo"ronych.
Regu"la 8, dotycz"aca typowania element"ow rekordu, jest jedn"a z regu"l,
w kt"orych "slad programu jest otrzymywany przez z"lo"renie "slad"ow podprogram"ow:
$$
\frac{\C\ts\exp\ra\tau,\U,\trace\qquad\langle\C\ts\labexps\ra\varrho,\U',\trace'\rangle}
     {\C\ts\longlabexps\ra\{\lab\mapsto\tau\}\langle +\
\varrho\rangle,
\ \U\langle\phantom{}\cup\U'\rangle,\ \trace\langle\append{}{\trace'}\rangle}
\eqno(8)
$$
Regu"la ta jest bardzo skomplikowana. Zamiast niej, 
za podstaw"e naszych rozwa"ra"n przyjmijmy, nie wyst"epuj"ac"a w definicji EML, 
regu"l"e typowania pary wyra"re"n, b"ed"acej uproszczeniem szczeg"olnego przypadku regu"ly 8:
$$
\frac{\C\ts\exp\ra\tau,\trace\qquad\C\ts\exp'\ra\tau',\trace'}
     {\C\ts(\exp,\exp')\ra\{1\mapsto\tau,2\mapsto\tau'\},\append{\trace}{\trace'}}
$$

\begin{obs}
\label{app_phase}
Niech program $T$ ma posta"c $(P, Q)$.
Niech $C$ b"edzie kontekstem. Wtedy elaboracja $T$ w kontek"scie $C$ ma dwie g"l"owne fazy:
\begin{itemize}
\item elaborowany jest podprogram $P$ w kontek"scie $C$, przy czym otrzymywane jest podstawienie
$\rea_P$ i "slad tej elaboracji $\trace_P$, 
\item elaborowany jest podprogram $Q$ w kontek"scie $\rea_P(C)$, przy czym otrzymywane jest podstawienie
$\rea_Q$ i "slad tej elaboracji $\trace_Q$. 
\end{itemize}
\end{obs}

Powstaje pytanie jak uzyska"c "slad elaboracji $T$ w kontek"scie $C$, 
zak"ladaj"ac "re $\trace_P$ i $\trace_Q$ s"a poprawnymi "sladami elaboracji $P$ i $Q$?
Narzucaj"aca si"e odpowied"z, "re "slad $T$ r"owny jest $\append{\trace_P}{\trace_Q}$,
niestety nie jest prawdziwa. "Slad $\trace_P$ uzyskiwany jest w potencjalnie 
nieg"l"ownym dla $T$ kontek"scie, wi"ec mimo i"r $\trace_P$ jest g"l"owny dla $P$, 
mo"re by"c zbyt og"olny, by sta"c si"e cz"e"sci"a "sladu $T$.
Ilustruje to nast"epuj"acy prosty przyk"lad:
\begin{verbatim}
fun g fib = (fib, fib 5)
\end{verbatim}
gdzie $P$ to \|fib|, za"s $Q$ to \|fib 5|.
W $\trace_P$ zmienna \|fib| nie ma przypisanego typu funkcyjnego, 
mimo "re w "sladzie $T$ zmienna \|fib| mo"re mie"c jedynie typ funkcyjny,
gdy"r z elaboracji $Q$ wynika, "re \|fib| jest funkcj"a 
(w szczeg"olno"sci $\rea_Q$ przypisuje zmiennej typowej, 
kt"ora jest typem \|fib|, typ funkcyjny).

\begin{thm}
\label{app_subst}
Przyjmijmy oznaczenia ze Spostrze"renia~\ref{app_phase}.
Wtedy "slad elaboracji $T$ w kontek"scie $C$ r"owny jest $\append{\widehat{\trace}}{\trace_Q}$,
gdzie $\widehat{\trace}$ jest "sladem elaboracji $P$ w kontek"scie $\rea_Q(\rea_P(C))$.
\end{thm}

\begin{proof}
Poniewa"r $\trace_Q$ jest "sladem elaboracji
$Q$ w kontek"scie $\rea_P(C)$, wi"ec z definicji "sladu elaboracji,
$\trace_Q$ jest "sladem analizy statycznej $Q$ w kontek"scie $\rea_Q(\rea_P(C))$.

Z drugiej strony wynikiem elaboracji $T$ w kontek"scie $C$ 
jest wywiedzenie g"l"ownego obiektu semantycznego
dla $T$ w kontek"scie $\rea_Q(\rea_P(C))$, 
wed"lug regu"ly typowania pary wyra"re"n, ale z pomini"eciem "slad"ow.
Ale w takim razie z regu"ly typowania pary wyra"re"n wynika teza.
\end{proof}

Odpowiedniki tego prostego twierdzenia dla innych regu"l, 
w kt"orych na\-st"e\-pu\-je z"lo"renie "slad"ow, s"a r"ownie"r do"s"c oczywiste.
Co wi"ecej podobne twierdzenia mo"rna udowodni"c r"ownie"r dla j"ezyka Modu"l"ow EML,
z tym "re zamiast podstawie"n b"ed"a w nich wyst"epowa"c realizacje.

Niestety bezpo"srednie zastosowanie w algorytmie zbierania "slad"ow
procedury jaka wynika z Twierdzenia~\ref{app_subst},
prowadzi do podw"ojnej elaboracji~$P$. Pierwszy raz w kontek"scie $C$, 
aby uzyska"c podstawienie $\rea_P$ i drugi raz w kontek"scie $\rea_Q(\rea_P(C))$,
aby uzyska"c "slad $\widehat{\trace}$.

\subsubsection{\secscheme}
\label{sec:scheme}

Jednym z najbardziej niepokoj"acyh zjawisk, wyst"epuj"acych
podczas budowania "sladu programu ze "slad"ow podprogram"ow, jest to, 
i"r "slady podprogram"ow czasami powstaj"a na skutek domykania
nieg"l"ownych dla ca"lego programu "slad"ow, wzgl"edem r"ownie"r nieg"l"ownych kontekst"ow.
Dow"od na\-st"e\-pu\-j"a\-ce\-go twierdzenia pokazuje, 
"re zjawisko to nie musi by"c gro"zne.
\begin{thm}
\label{rea_clos}
Niech $P$ b"edzie programem napisanym w j"ezyku J"adra Extended ML.
Niech $D$ b"edzie kontekstem.
Niech $\trace_P$ b"edzie "sladem elaboracji $P$ w kontek"scie $D$.
Niech $\rea$ b"edzie podstawieniem.

Wtedy "slad elaboracji $P$ w kontek"scie $\rea(D)$ r"owny jest $\rea(\trace_P)$.
\end{thm}
 
\begin{proof}
Indukcja po strukturze $\trace_P$.
Przypadki bazowe i przypadek sk"ladania "slad"ow wynikaj"a z podstawowych w"lasno"sci procesu elaboracji.
Pozostaje przypadek, gdy $\trace_P \in \TraceScheme$.
Wtedy, w ostatnim kroku elaboracji $P$, musia"l wyst"epowa"c pewien "slad $\trace$, 
domykany wzgl"edem pewnego kontekstu $C$, przy czym $\trace_P = \cl{C}{\trace}$.

Niech $P'$ b"edzie podprogramem $P$ odpowiadaj"acym $\trace$.
Dla uproszczenia za"l"o"rmy, "re $P$ jest postaci $\VAL\ P'$,
a wi"ec elaboracja $P'$ jako cz"e"sci $P$ odby"la si"e w kontek"scie $D$.
Niech ta elaboracja daje podstawienie $\rea_{P'}$.
W takim razie z Twierdzenia~\ref{clos_traces} wiemy, "re $C = \rea_{P'}(D)$.

W razie potrzeby mo"rna przemianowa"c w $\trace$, $P'$ i $P$
zmienne typowe wyst"epuj"ace wolno w $\trace$, a nie wyst"epuj"ace wolno w $C$ tak, 
aby znalaz"ly si"e one poza zasi"egiem $\rea$.
Z uzyskanej z dalszego rozumowania tezy, "re dla przemianowanego $P$ i $\trace$, 
"slad elaboracji $P$ w kontek"scie $\rea(D)$ r"owny jest $\rea(\cl{C}{\trace})$,
"latwo dosta"c tez"e dla oryginalnych $P$ i $\trace$.

Stosuj"ac hipotez"e indukcyjn"a do $P'$, $D$, $\trace$ i $\rea$ dostaj"e,
"re "slad elaboracji $P'$ w kontek"scie $\rea(D)$ r"owny jest $\rea(\trace)$.
Niech ta elaboracja daje podstawienie $\phi_{P'}$.
Z Twierdzenia~\ref{clos_traces} otrzymuj"e, "re "slad elaboracji $P$ 
w kontek"scie $\rea(D)$ r"owny jest $\cl{\phi_{P'}(\rea(D))}{\rea(\trace)}$. Ale mo"rna pokaza"c, "re
$$
\cl{\phi_{P'}(\rea(D))}{\rea(\trace)} = \cl{\rea(\rea_{P'}(D))}{\rea(\trace)} = \cl{\rea(C)}{\rea(\trace)}
$$
Pierwsz"a z tych r"owno"sci uzyskuje si"e przez "rmudn"a indukcj"e po strukturze $P'$,
korzystaj"ac z faktu "re odpowiednie zmienne typowe znajduj"a si"e poza zasi"egiem $\rea$
(by"c mo"re na skutek przemianowania zmiennych), druga wynika z tego, "re $C = \rea_{P'}(D)$.

Pozostaje udowodni"c, "re 
$$\cl{\rea(C)}{\rea(\trace)} = \rea(\cl{C}{\trace})$$
Dla uproszczenia za"lo"r"e do ko"nca dowodu, 
i"r w $\trace$ wyst"epuje dok"ladnie jedna typowa zmienna wolna --- $\alpha$.
%Przy u"ryciu do"s"c rozbudowanej buchalterii mo"rna 
"Latwo to rozumowanie przenie"s"c 
na przypadki z inn"a liczb"a zmiennych.

Rozpatrz"e nast"epuj"ace mo"rliwo"sci:
\begin{itemize}
\item $\alpha$ wyst"epuje wolno w kontek"scie $C$.
      Wtedy $\cl{C}{\trace} = \trace$, a~st"ad
      $$\rea(\cl{C}{\trace}) = \rea(\trace)$$
      Z drugiej za"s strony jedyne zmienne wolne $\rea(\trace)$ 
      to zmienne wolne wyst"epuj"ace w $\rea(\alpha)$,
      co wraz z faktem, i"r w $C$ wyst"epuje wolno $\alpha$, 
      czyli w $\rea(C)$ wyst"epuj"a wolno zmienne wolne termu $\rea(\alpha)$, 
      daje $$\cl{\rea(C)}{\rea(\trace)} = \rea(\trace)$$
\item $\alpha$ nie wyst"epuje wolno w kontek"scie $C$.
      Wtedy $\cl{C}{\trace} = \forall\alpha.\trace$ a~poniewa"r
      $\forall\alpha.\trace$ nie posiada zmiennych wolnych, dostaj"e      
      $$\rea(\cl{C}{\trace}) = \forall\alpha.\trace$$ 
      Z drugiej strony poniewa"r $\alpha$ jest poza zasi"egiem $\rea$, 
      to $\rea(\trace) = \trace$ oraz $\rea(C)$ nie zawiera $\alpha$, a st"ad
      $$\cl{\rea(C)}{\rea(\trace)} = \forall\alpha.\trace$$
\end{itemize}
\end{proof}

Teraz wreszcie mo"remy pokaza"c, w jaki spos"ob z poprawnych "slad"ow elaboracji podprogram"ow
mo"rna zbudowa"c "slad elaboracji ich pary.
\begin{thm}
\label{app}
Niech program $T$ ma posta"c $(P, Q)$.
Niech $C$ b"edzie kontekstem w kt"orym przeprowadzana jest e\-la\-bo\-ra\-cja $T$.
Niech $\rea_P$ b"edzie podstawieniem, wynik"lym z elaboracji $P$ jako cz"e"sci $T$,
i niech $\trace_P$ b"edzie "sladem tej elaboracji. 
(Zauwa"rmy, "re $P$ jest elaborowane w kontek"scie $C$.)
Niech $\rea_Q$ b"edzie podstawieniem, wynik"lym z elaboracji $Q$ jako cz"e"sci $T$,
i niech $\trace_Q$ b"edzie "sladem tej elaboracji. 

Wtedy $\append{\rea_Q(\trace_P)}{\trace_Q}$ jest "sladem elaboracji $T$.
\end{thm}
 
\begin{proof}
Z Twierdzenia ~\ref{app_subst} wiemy, i"r "slad elaboracji $T$ w kontek"scie $C$ 
r"owny jest $\append{\widehat{\trace}}{\trace_Q}$,
gdzie $\widehat{\trace}$ jest "sladem elaboracji $P$ w kontek"scie $\rea_Q(\rea_P(C))$.

Zastosujmy Twierdzenie~\ref{rea_clos} z $\rea = \rea_P$ i $D = C$.
Dostajemy, "re "sladem elaboracji $P$ w kontek"scie $\rea_P(C)$ jest $\rea_P(\trace_P)$.
Ale poniewa"r $\rea_P$ zawiera tylko te konkretyzacje typ"ow,
kt"ore s"a niezb"edne by uda"la si"e elaboracja $P$ w $C$ (por"ownaj rozdzia"l~\ref{sec:elabexp}),
a $\trace_P$ jest "sladem tej w"la"snie elaboracji, mamy $\rea_P(\trace_P) = \trace_P$.
A wi"ec "slad elaboracji $P$ w kontek"scie $\rea_P(C)$ r"owny jest $\trace_P$. 

Je"sli teraz postawimy $\rea = \rea_Q$ i $D = \rea_P(C)$ to ponownie spe"lnione 
zostan"a za"lo"renia Twierdzenia~\ref{rea_clos}, i uzyskamy
$\widehat{\trace} = \rea(\trace_P) = \rea_Q(\trace_P)$.
\end{proof}

Podobne twierdzenia mo"rna udowodni"c dla wszystich postaci program"ow J"adra EML,
w kt"orych sk"ladane s"a "slady i mo"re wyst"epowa"c zjawisko domykania 
niedoprecyzowanego "sladu, wzgl"edem nieg"l"ownego kontekstu.
Co wi"ecej, twierdzenia podobne do~\ref{rea_clos} i~\ref{app} mo"rna udowodni"c
r"ownie"r w przypadku j"ezyka Modu"l"ow, gdy rol"e podstawie"n spe"lniaj"a realizacje.

\begin{cor}
\label{single_pass}
Zbierania "slad"ow mo"rna dokonywa"c wraz z elaboracj"a, przy pomocy jedoprzebiegowego algorytmu.
\end{cor}

%%% Local Variables: 
%%% mode: latex
%%% TeX-master: "eml-type"
%%% End: 
