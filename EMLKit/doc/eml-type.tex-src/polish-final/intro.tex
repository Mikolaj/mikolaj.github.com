
\section{\secover}
\label{sec:over}

W ostatnich latach programowanie funkcyjne i metody specyfikacji,
czerpi"ac z~siebie nawzajem, szybko si"e rozwija"ly.
Owocami tego po"rytecznego zbli\-"re\-nia s"a mi"edzy innymi formalizm Extended ML,
a w dalszym rz"edzie system EML Kit.
W tym rozdziale postaram si"e je opisa"c 
na tle "srodowiska, w kt"orym si"e uformowa"ly.

\subsection{\secsml}
\label{sec:sml}

Standard ML (SML)~\cite{MTH90} jest funkcyjnym j"ezykiem programowania,
co oznacza, "re dzia"lanie programu nie polega na stopniowej zmianie stanu maszyny,
jak to jest w przypadku j"ezyk"ow imperatywnych,
lecz na obliczaniu warto"sci wyra"re"n. 
Do opisywania wyra"re"n s"lu"ry cz"e"s"c Standard ML
zwana j"ezykiem J"adra.

Matematyczn"a podstaw"a modelu oblicze"n SML jest 
rachunek $\lambda$~\cite{Bar84} wraz z rozszerzonym systemem typ"ow ML~\cite{DM82}.
Uczynienie aplikowania funkcji g"l"ownym motorem ewaluacji
u"latwia "`matematyczne'' my"slenie o dzia"laniu programu.
Z kolei silny system typ"ow SML, dzi"eki swojej elastyczno"sci i og"olno"sci
nie kr"epuj"ac u"rytkownika, pozwala wykrywa"c
zdecydowan"a wi"ek\-szo"s"c b"l"ed"ow jeszcze przed wykonaniem programu,
podczas procesu zwanego analiz"a statyczn"a.

SML posiada bardzo silne mechanizmy modularyzacji,
do kt"orych mo"rna si"e odwo"lywa"c przy pomocy j"ezyka Modu"l"ow.
Do budowania globalnej struktury programu s"lu"r"a struktury,
kt"ore s"a atomowymi modu"lami, i funktory, wyra"raj"ace hierarchiczne 
zale"rno"sci mi"edzy modu"lami i s"lu"r"ace do ich sk"la\-da\-nia.

Podstawowym narz"edziem do opisu w"lasno"sci modu"l"ow, 
czyli do ich specyfikacji, s"a sygnatury:

\begin{verbatim}
signature TOTAL_PREORDER =
sig
    type elem 
   val leq : elem * elem -> bool 
    (* leq should be a total preorder *)
end
\end{verbatim}
\|TOTAL_PREORDER| to nazwa sygnatury, zdanie "`\|type elem|'' postuluje istnienie
typu o nazwie elem, za"s zdanie "`\|val leq : elem * elem -> bool|'' stwierdza, 
"re powinna istnie"c funkcja \|leq| o typie \|elem * elem -> bool|.
Komentarz w j"ezyku naturalnym zamykaj"acy sygnatur"e opisuje "r"adane w"las\-no\-"sci funkcji \|leq|.

Zar"owno struktury jak i funktory mog"a by"c opatrzone sygnaturami.
Ta przyk"ladowa struktura jest zadeklarowana jako pasuj"aca 
do zdefiniowanej wcze"sniej sygnatury \|TOTAL_PREORDER|:
\begin{verbatim}
structure OddRational : TOTAL_PREORDER =
struct
    type elem = int * int
    fun leq ((m, n), (k, l)) = 
        if n = 0 andalso l = 0 
            then m < 0 orelse (m > 0 andalso k > 0)
        else if (n < 0 andalso l >= 0) 
                orelse (n >= 0 andalso l < 0)
                 then k * n <= m * l
             else m * l <= k * n 
end
\end{verbatim}
Typ \|elem| jest tu par"a liczb ca"lkowitych,
za"s funkcja \|leq| ma pewne cechy naturalnego porz"adku na liczbach wymiernych.
Mo"rna pokaza"c, "re wszystkie wymogi wyra"rone w \|TOTAL_PREORDER| zosta"ly spe"lnione
(r"ownie"r te opisane w komentarzu).

\subsection{\seceml}
\label{sec:eml}

Po"l"aczenie prostoty modelu oblicze"n J"adra
z si"l"a j"ezyka Modu"l"ow czyni z SML wspania"le narz"edzie 
do tworzenia i piel"egnowania du"rych i z"lo"ronych system"ow oprogramowania.
Narzucaj"acymi si"e sposobami pracy z SML s"a warianty metodologii specyfikacji algebraicznych, 
cho"cby takie w kt"orych u"rywa si"e j"ezyka naturalnego, jak w przyk"ladzie powy"rej.

Dzi"eki prostocie J"adra SML istnieje du"ra swoboda wyboru 
j"ezyk"ow do wyra"rania w"laso"sci program"ow. 
Mo"rna u"rywa"c j"ezyk"ow bardzo s"labych, jak na przyk"lad logika r"owno"sciowa, 
kt"ore w wypadku programowania imperatywnego nie mia"lyby szans zastosowania.
Mo"rna u"rywa"c tak"re j"ezyk"ow silnych, 
pozwalaj"acych wyra"ra"c skomplikowane w"lasno"sci w spos"ob bardzo zwarty.
Do takich j"ezyk"ow nale"ry na przyk"lad bardzo mocne rozszerzenie logiki pierwszego rz"edu, 
za pomoc"a  kt"orego buduje si"e cia"la aksjomat"ow w Extended~ML.

Extended ML (EML)~\cite{bib:KST94} jest formalizmem do wywodzenia program"ow w Standard ML
z ich specyfikacji. Proces konstrukcji oprogramowania odbywa si"e stopniowo,
pocz"awszy od czystej specyfikacji, poprzez stadia, w kt"orych przeplataj"a
si"e fragmenty specyfikacji i kodu w SML, a sko"nczywszy na w~pe"lni wykonywalnym programie w SML.
Wszystkie kroki wyra"rane s"a w Extended ML, a ich poprawno"s"c dowodzona jest
wed"lug regu"l Semantyki Weryfikacyjnej EML.

Jako j"ezyk, Extended ML jest rozszerzeniem (du"rego podzbioru) Standard ML.
Mo"rna go podzieli"c, podobnie jak w przypadku SML, na j"ezyk J"adra i j"ezyk Modu"l"ow.
Szczeg"oln"a rol"e w procesie tworzenia oprogramowania odgrywa system modu"l"ow EML
dopuszczaj"acy, opr"ocz element"ow znanych z SML, sygnatury w kt"orych wyst"epuj"a aksjomaty:

\begin{verbatim}
signature TOTAL_PREORDER =
sig
    type elem 
    val leq : elem * elem -> bool 
    axiom forall (a, b) => leq (a, b) orelse leq (b, a)
    and forall (c, d, e) => 
        (leq (c, d) andalso leq (d, e)) implies leq (c, e)
end
\end{verbatim}

Dzi"eki swojemu ca"lkowitemu sformalizowaniu oraz sile mechanizm"ow modularyzacyjnych
i j"ezyka specyfikacji, Extended ML daje mo"rliwo"s"c konstruowania oprogramowania
o niezwyk"lej niezawodno"sci i "latwo"sci rozbudowy oraz piel"egnacji.
O wygodzie stosowania EML decyduj"a dwie rzeczy: czytelno"s"c i klarowno"s"c
formalnej definicji oraz dost"epno"s"c i jako"s"c narz"edzi wspomagaj"acych.

\subsection{\secformal}
\label{sec:formal}

J"ezyk Standard ML jest "sci"sle opisany w swojej definicji \cite{MTH90}.
Opis semantyki sk"lada si"e z dw"och g"l"ownych cz"e"sci: Semantyki Statycznej i Semantyki Dynamicznej.
Ka"rda z nich podzielona jest z kolei na fragment dotycz"acy j"ezyka J"adra
oraz fragment dotycz"acy j"ezyka Modu"l"ow.

Formalizm wykorzystany do przedstawienia semantyki SML zwie si"e semantyk"a naturaln"a~\cite{Han93}.
W sk"lad definicji wchodz"a zbiory regu"l postaci:
$$
\frac{\begin{array}{c}
\nu_1\ \ \ \cdots\ \ \ \nu_k
\end{array}}
     {\phi}
$$ %$$
gdzie konkluzja $\phi$ jest zdaniem, za"s przes"lanki $\nu_i$ s"a albo zdaniami albo regu"lami.
W przypadku Semantyki Statycznej zdania m"owi"a o mo"rliwo"sci przypisania elementowi j"ezyka
pewnego typu, w przypadku Semantyki Dynamicznej --- warto"sci.

Extended ML r"ownie"r posiada formaln"a definicj"e \cite{bib:KST94}.
Podobnie jak EML jest w pewnym sensie rozszerzeniem SML,
tak i definicja EML jest niejako nadbudowana na definicji SML. 
Cho"c miejscami zmodyfikowany jest spos"ob prezentacji,
wprowadzone nowe elementy, poprawione b"l"edy,
to jednak og"olna struktura odziedziczona z definicji SML pozostaje zachowana.
Jedyn"a zupe"lnie now"a cz"e"sci"a definicji EML jest 
Semantyka Weryfikacyjna, opisana tym samym formalizmem co Semantyka Statyczna i Semantyka Dynamiczna, 
a s"lu"r"aca wyra"reniu w"lasno"sci spe"lnienia specyfikacji przez implementacj"e.

Semantyka naturalna czyni mo"rliwym opisanie tak wielkiego
i skomplikowanego formalizmu jak EML we wzgl"ednie zwarty i elegancki spos"ob.
Ta forma przezentacji ma jednak r"ownie"r swoje z"le strony.
Niekt"ore drobne b"l"edy w definicji potrafi"a d"lugo pozostawa"c niezauwa"rone.
Niekt"ore mechanizmy wydaj"a si"e proste i jednoznaczne jedynie dot"ad,
dop"oki nie zostanie podj"eta pierwsza pr"oba ich implementacji.
Niekt"ore globalne relacje pomi"edzy cz"e"sciami definicji
s"a niejasne, dop"oki jej hierarchia modu"l"ow nie zostanie
opisana w powa"rnym formalizmie do definiowania struktury modularnej
takim jak na przyk"lad j"ezyk Modu"l"ow SML.

\subsection{\seckit}
\label{sec:kit}

O systemie EML Kit mo"rna my"sle"c (z pewn"a doz"a dobrej woli)
jako o w pe"lni deterministycznym, algorytmicznym i szczeg"o"lowym
opisie Semantyki Statycznej i Semantyki Dynamicznej EML.

EML Kit zmierza do tego, by pe"lni"c funkcje (podstawy i miejca do testowania)
wygodnego "srodowiska do formalnego wywodzenia program"ow,
przy u"ryciu j"ezyka, formalizmu i metodologii Extended ML.

Wersja 1.0 stworzona przez Marcina Jurdzi"nskiego, Miko"laja Konarskiego, 
S"lawomira Leszczy"nskiego i Aleksego Schuberta,
jest kompletn"a implementacj"a Extended ML jako "`j"ezyka programowania''.
Pozwala ona dokonywa"c analizy sk"ladniowej, sprawdza"c poprawno"s"c typow"a
i wylicza"c warto"s"c dowolnych program"ow EML, czy b"edzie to czysty Standard ML,
czy te"r pe"lny EML z aksjomatami i innymi konstrukcjami specyfikacyjnymi.
Proces ten odbywa si"e w "scis"lej zgodno"sci z Semantyk"a Statyczn"a i Sematyk"a Dynamiczn"a EML.

EML Kit bazuje na systemie ML Kit~\cite{BRTT93};
elastycznym, eleganckim i zmodularyzowanym interpreterze Standard ML, napisanym w Standard ML.
ML~Kit jest wiern"a i tak bezpo"sredni"a, jak to mo"rliwe,
implemetacj"a j"ezyka zdefiniowanego w definicji SML.
Sk"ladnia abstakcyjna jest niemal dos"lownie wzi"eta z definicji,
a projekt wielu innych detali, takich jak na przyk"lad konwencje nazewnicze,
jest "sci"sle zainspirowany definicj"a.

\subsection{\secdepend}
\label{sec:depend}

Dwa fakty --- ten, "re definicja EML jest rozszerzeniem definicji SML,
i ten, "re ML Kit jest tak bliski tej ostatniej --- pomog"ly nam
w naszych wysi"lkach uczynienia systemu EML Kit wiern"a i klarown"a
implementacj"a Extended~ML. Byli"smy w stanie rozszerzy"c ML Kit
analogicznie do sposobu w jaki definicja EML rozszerza definicj"e SML,
zachowuj"ac styl programowania w"la"sciwy dla systemu ML Kit.
W ten spos"ob stworzyli"smy podstaw"e do rozwoju przysz"lych
narz"edzi "Srodowiska Programistycznego Extended ML,
takich jak generatory musik"ow dowodowych, czy prototypowe systemy dowodzenia.
Ilustruje to znajomo wygl"adaj"acy diagram poni"rej.

\vfil
\vspace{10pt}
\begin{center}
{\footnotesize
\setlength{\unitlength}{6.7cm}
\begin{picture}(1.4,2.1)

\put(0.7,2.1){\makebox(0,0){
        \begin{tabular}{c}
        Programistyczne\\
        "Srodowisko EML
        \end{tabular}}}

\put(0.7,1.4){\makebox(0,0){
        \begin{tabular}{c}
        EML Kit
        \end{tabular}}}
\put(0.7,1.47){\line(0,1){0.07}}
\put(0.7,1.58){\line(0,1){0.07}}
\put(0.7,1.69){\line(0,1){0.07}}
\put(0.7,1.80){\line(0,1){0.07}}
\put(0.7,1.91){\vector(0,1){0.10}}

\put(0.0,0.7){\makebox(0,0){
        \begin{tabular}{c}
        ML Kit
        \end{tabular}}}
\put(0.07,0.77){\vector(1,1){0.566}}

\put(1.4,0.7){\makebox(0,0){
        \begin{tabular}{c}
        Definicja EML
        \end{tabular}}}
\put(1.33,0.77){\vector(-1,1){0.566}}


\put(0.7,0.0){\makebox(0,0){
        \begin{tabular}{c}
        Definicja SML
        \end{tabular}}}
\put(0.77,0.07){\vector(1,1){0.566}}
\put(0.63,0.07){\vector(-1,1){0.566}}

\end{picture}

}
\end{center}
\vfil

\section{\secanal}
\label{sec:anal}

Ka"rdy nowoczesny j"ezyk programowania oparty jest na w"la"sciwym sobie
prostym, klarownym matematycznym modelu.
Proces analizy statycznej programu, przebiegaj"acy mi"edzy jego napisaniem
a wykonaniem, polega na przetworzeniu go w odpowiadaj"acy mu matematyczny obiekt.
W tym rozdziale chcia"lbym przedstawi"c bli"rej to uj"ecie analizy statycznej,
ilustruj"ac je konkretnymi przyk"ladami dotycz"acymi specyfiki Extended ML.

\subsection{\secall}
\label{sec:all}

W sk"lad analizy statycznej wchodz"a dzia"lania, 
dokonywane w z g"ory ograniczonym czasie, a maj"ace za podstaw"e tekst programu.
Wynika st"ad, "re w przypadku j"ezyk"ow dopuszczaj"acych zjawisko nieterminacji,
samego procesu wykonywania programu nie mo"rna zaliczy"c do analizy statycznej.

Analiz"e statyczn"a wykonuje si"e, by sprawdzi"c poprawno"s"c program"ow,
na przyk"lad poprawno"s"c sk"ladniow"a, oraz by przygotowa"c program do wykonania,
by"c mo"re poddaj"ac go wyrafinowanym przekszta"lceniom w celu optymalizacji.
Je"sli potraktujemy Extended ML jako j"ezyk programowania,
to na podstawie systemu EML Kit b"edziemy mogli obejrze"c
elementy przyk"ladowej analizy statycznej tego j"ezyka.

Pierwszym stadium analizy statycznej w systemie EML Kit jest analiza sk"ladniowa, 
to znaczy przypisanie programowi jego drzewa rozbioru wed"lug gramatyki podanej w definicji Extended ML.
Potem nast"epuje szereg dodatkowych korekcji tego drzewa,
wynikaj"acych ze specyfiki sk"ladni EML i sposobu jej prezentacji,
a w rezultacie powstaje ostateczne drzewo sk"ladni abstrakcyjnej programu,
kt"ore b"edzie u"ryte jako podstawa zar"owno do dalszej 
analizy statycznej jak i procesu wykonywania.
Nast"epnym etapem jest elaboracja, to znaczy sprawdzanie 
poprawno"sci programu je"sli chodzi o typy.
Elaboracja odbywa si"e w oparciu o Semanyk"e Statyczn"a J"adra EML
i Semantyk"e Statyczn"a Modu"l"ow EML, a przy przej"sciu z j"ezyka
J"adra do Modu"l"ow dodatkowo odbywa si"e proces zwany
rozwik"lywaniem prze"ladowania \ang{overloading resolution}.
 
Poniewa"r programy napisane w EML mog"a si"e p"etli"c,
ich wykonywanie, kt"ore powinno si"e odbywa"c na podstawie
Semantyki Dynamicznej EML, nie mo"re by"c zaliczone do analizy statycznej.
Z podobnych powod"ow weryfikacja program"ow w EML,
kt"ora odbywa"laby si"e na podstawie Semantyki Weryfikacyjnej EML,
nie mo"re by"c zaliczona do analizy statycznej.
Ale ju"r generowanie musik"ow dowodowych \ang{proof obligations} ---
pewna operacja wst"epna wzgl"edem weryfikacji, 
kt"ora by"c mo"re zostanie w przysz"lo"sci
zaimplementowana w ramach systemu EML Kit --- 
zdecydowanie ma charakter statyczny i mo"rna j"a przytoczy"c jako przyk"lad
analizy statycznej, cho"c mo"re nieco wy"rszego, ni"r si"e to zwykle spotyka, rz"edu.

W pozosta"lej cz"e"sci rozdzia"lu (i w ca"lej pracy) 
skupimy si"e na przypadku analizy statycznej w najw"e"rszym
rozumieniu tego terminu, to znaczy na sprawdzania poprawno"sci typowej 
fraz j"ezykowych, wzgl"edem zadanej sematyki statycznej.

\subsection{\secbasic}
\label{sec:basic}

\subsubsection{\secsemantic}
\label{sec:semantic}

Sematyka statyczna frazom j"ezyka, czyli obiektom syntaktycznym, 
przypisuje pewne twory matematyczne, zwane obiektami semantycznymi.
Pe"lne zestawienie obiekt"ow semantycznych Semantyki Statycznej EML
znajduje si"e w dodatku~\ref{sec:semobjeml}.

Najbardziej oczywistym przyk"ladem obiektu semantycznego jest typ.
W Semantyce Statycznej J"adra EML wyra"reniom przypisywane s"a
obiekty semantyczne $\tau$, nale"r"ace do dziedziny semantycznej typ"ow $\Type$.
Wyra"reniom b"e\-d"a\-cy\-mi rekordami przypisywane s"a typy rekordowe,
wyra"reniom reprezentuj"acym funkcje przypisywane s"a typy funkcyjne, itd. 

Innym powszechnie spotykanym rodzajem obiekt"ow semantycznych s"a "srodowiska.
Reprezentuj"a one sko"nczony zbi"or nazw, wraz z przypisanymi nazwom
obiektami semantycznymi, na przyk"lad typami.
W Extended ML wynikiem analizy statycznej deklaracji s"a 
obiekty semantyczne oznaczane~$\E$, 
nale"r"ace do dziedziny semantycznej "srodowisk $\Env$.

W Semantyce Statycznej Modu"l"ow wyst"epuj"a jeszcze bardziej z"lo"rone
obiekty semantyczne. Strukturom odpowiadaj"a semantyczne struktury,
sk"la\-da\-j"a\-ce si"e z nazwy struktury i "srodowiska opisuj"acego
komponenty struktury, wraz z odpowiadaj"acymi im obiektami semantycznymi:
\begin{displaymath}
\begin{array}{rcl}
\S\ {\rm lub}\ (\m,\E)
                & \in   & \Str = \StrNames\times\Env \\
\end{array}
\end{displaymath}
gdzie $S$ to meta-zmienna przebiegaj"aca dziedzin"e $\Str$, 
b"ed"ac"a produktem dziedzin $\StrNames$ i $\Env$, 
po kt"orych przebiegaj"a z kolei meta-zmienne $\m$~i~$\E$.

Sygnaturom odpowiadaj"a semantyczne sygnatury,
sk"ladaj"ace si"e z semantycznej struktury i zbioru nazw,
o kt"orych nale"ry my"sle"c, "re s"a zwi"azane:
\begin{displaymath}
\begin{array}{rcl}
\sig\ {\rm lub}\ \longsig{}
                & \in   & \Sig =  \NameSets\times\Str \\
\end{array}
\end{displaymath}
gdzie $\sig$, $N$ i $S$ to znowu meta-zmienne.

\subsubsection{\secjudgments}
\label{sec:judgments}

Os"ady to zdania postaci:
\[C\ts{\it fraza}\ra A\]
gdzie $C$ i $A$ to obiekty semantyczne, za"s {\it fraza} to element j"ezyka.
Zdanie takie mo"rna czyta"c "`w kontek"scie $C$, {\it fraza} daje $A$''.

Przy pomocy Semantyki Statycznej J"adra EML mo"rna wywodzi"c na przy\-k"lad zdania
podobne do:
$$\E\ts\exp\ra\tau$$
("`w "srodowisku $\E$, wyra"renie $\exp$ daje typ $\tau$''),
$$\E_1\ts\dec\ra\E_2$$
("`w "srodowisku $\E_1$, deklaracja $\dec$ daje "srodowisko $\E_2$'').
%gdzie $\exp$ jest wyra"reniem, a $\dec$ deklaracj"a.
Za"s w Semantyce Statycznej Modu"l"ow pojawiaj"a si"e mi"edzy innymi os"ady postaci:
$$\B\ts\sigexp\ra\S,\trace$$
gdzie $\B$ to baza zawieraj"aca w szczeg"olno"sci "srodowisko, $\sigexp$ to sygnatura, a $\trace$ to "slad.

Aby zasygnalizowa"c, czym jest "slad, musimy przypomie"c, 
"re w Semantyce Statycznej EML os"adami prawdziwymi s"a te, kt"ore mo"rna wywie"s"c przy pomocy regu"l.
Ka"rdy prawdziwy os"ad posiada wi"ec (by"c mo"re wi"ecej ni"r jedno) drzewo wywodu.
W pewnym uproszczeniu "slady s"a w"la"snie drzewami wywodu pewnych os"ad"ow.


\subsubsection{\secprincipal}
\label{sec:principal}

Niech $\statsem$ b"edzie semantyk"a statyczn"a 
i niech $\succ$ b"edzie cz"e"sciowym porz"adkiem na zbiorze wywod"ow $\statsem$.
Wtedy obiekt semantyczny $A$ nazywamy g"l"ownym \ang{principal} dla obiektu syntaktycznego ${\it fraza}$
w kontek"scie obiektu semantycznego $C$, wzgl"edem relacji $\succ$ , 
w ramach semantyki $\statsem$, gdy
\begin{itemize}
\item istnieje wyw"od $\trace$ w ramach $\statsem$ os"adu $\C\vdash{\it fraza}\ra A$, 
\item je"sli $\trace'$ jest wywodem $\C\vdash{\it fraza}\ra A'$ w ramach $\statsem$, 
      to $\trace\succ\trace'$.
\end{itemize}

W Semantyce Statycznej EML pojawiaj"a si"e definicje maj"ace charakter definicji obiektu g"l"ownego.
Przytocz"e dwie spo"sr"od nich.

Typ $\tau$ wraz ze "sladem $\trace$ jest g"l"owny dla wyra"renia $\exp$ w kontek"scie $C$,
gdy (w troch"e innym sformu"lowaniu ni"r oryginalne)
\begin{itemize}
\item $\C\vdash\exp\ra\tau,\trace$, 
\item je"sli $\C\vdash\exp\ra\tau',\trace'$, to
      $\tau\succ\tau'$ i $\trace\succ\trace'$.
\end{itemize} 
\pagebreak
gdzie "`$\succ$'' jest relacj"a uog"olniania typ"ow 
($\tau\succ\tau'$, gdy typ $\tau$ jest og"olniejszy ni"r $\tau'$,
czyli gdy typ $\tau'$ jest instancj"a typu $\tau$),
kt"or"a mo"rna w naturalny spos"ob rozszerzy"c na "slady.
Mo"rna udowodni"c, "re w SML (a tak"re w EML) ka"rde wyra"renie, 
kt"ore ma typ, ma r"ownie"r typ g"l"owny~\cite{Ler92}. 

M"owimy, "re sygnatura semantyczna $(\N)\S$ z $\trace$ jest g"l"owna 
dla sygnatury syntaktycznej $\sigexp$ w bazie $\B$ 
gdy (w uproszczeniu)
\begin{itemize}
\item $\B\vdash\sigexp\ra\S,\trace$, 
\item je"sli $\B\vdash\sigexp\ra\S',\trace'$, to
      $\sigord{\longsig{}}{}{\S'}$ i $\trace\succ\trace'$.
\end{itemize}
gdzie relacja "`$\sigord{}{}$'' jest instancjacj"a sygnatur.
W dodatku A do ksi"a"rki~\cite{MT91} zawarty jest dow"od
istnienia sygnatur g"l"ownych w SML. Mo"rna pokaza"c,
"re r"ownie"r w EML ka"rda sygnatura, dla kt"orej mo"rna
wywie"s"c sygnatur"e semantyczn"a, posiada sygnatur"e semantyczn"a g"l"own"a.

\subsection{\secinterest}
\label{sec:interest}

\subsubsection{\secindeterm}
\label{sec:indeterm}

Semantyk"e statyczn"a $\statsem$ nazywamy niedeterministyczn"a, 
je"sli istnieje taki element j"ezyka {\it fraza} i takie obiekty semantyczne $C$, $A_1$, $A_2$,
"re zar"owno os"ad $C\ts{\it fraza}\ra A_1$ jak i $C\ts{\it fraza}\ra A_2$
s"a wyprowadzalne w $\statsem$.

Sematyka Statyczna EML jest niedeterministyczna.
St"ad na przyk"lad nie ka"rdy typ, jaki mo"rna wywie"s"c dla wyra"renia, jest jego typem g"l"ownym.
W pustym kontek"scie wyra"reniu:
\begin{verbatim}
let
   fun S x y z = (x z) (y z)
in
   S
end 
\end{verbatim}     
mo"rna przypisa"c zar"owno typ 
$(\alpha\rightarrow\alpha\rightarrow\gamma)\rightarrow 
(\alpha\rightarrow\alpha)\rightarrow\alpha\rightarrow\gamma$, 
jak i typ 
$(\alpha\rightarrow\beta\rightarrow\beta)\rightarrow
(\alpha\rightarrow\beta)\rightarrow\alpha\rightarrow\beta$,    
ale "raden z nich nie jest r"owny g"l"ownemu typowi tego wyra"renia:
$(\alpha\rightarrow\beta\rightarrow\gamma)\rightarrow
(\alpha\rightarrow\beta)\rightarrow\alpha\rightarrow\gamma$.

\subsubsection{\secimposing}
\label{sec:imposing}

Ustalmy semantyk"e statyczn"a $\statsem$ i operacj"e syntaktyczn"a $insert$,
kt"ora w pewien spos"ob z dw"och obiekt"ow syntaktycznych tworzy jeden.
Niech $T$ i $P$ b"ed"a obiektami syntaktycznymi.
Niech $C$ b"edzie obiektem semantycznym 
i niech $\phi$ b"edzie w"lasno"sci"a obiekt"ow semantycznych.

Wtedy m"owimy, "re $P$ wymusza $\phi$ w $insert(P, T)$ przy $C$, gdy jednocze"snie
\begin{itemize}
\item istnieje obiekt semantyczny~$A$ taki, "re mo"rna wywie"s"c 
      $\C\vdash T \ra A$ oraz nie zachodzi w"lasno"s"c~$\phi$ dla~$A$,
\item dla ka"rdego~$A'$, dla kt"orego mo"rna wywie"s"c 
      $\C\vdash insert(P, T) \ra A'$, zachodzi w"lasno"s"c~$\phi$.
\end{itemize}

Wa"rnym przypadkiem wymuszania w Semantyce Statycznej EML
jest wymuszenie przez aksjomat zr"ownania typ"ow w sygnaturze.
Ilustruje to nast"epuj"acy przyk"lad:
\begin{verbatim}
sig
    type t
    type u
    val f : t -> u
    axiom forall x => f x == f (f x)
end
\end{verbatim}
Niech $insert(P, T)$ b"edzie powy"rsz"a sygnatur"a, 
$P$ wyst"epuj"acym w niej aksjomatem, a $T$ sygnatur"a bez aksjomatu.
Niech $C$ b"edzie pust"a baz"a, a $\phi$ w"lasno"sci"a obiektu semantycznego,
sk"ladaj"acego si"e w szczeg"olno"sci ze "sro\-do\-wis\-ka,
m"owi"ac"a, "re powinno ono przypisywa"c 
typom syntaktycznym \|t| i \|u| ten sam typ semantyczny.

Wida"c, "re obiekt powsta"ly w wyniku g"l"ownego wywodu dla $T$
nie spe"lnia w"lasno"sci $\phi$.
Natomiast obiekt powsta"ly w wyniku g"l"ownego wywodu dla sygnatury $insert(P, T)$,
a co za tym idzie r"ownie"r ka"rdy obiekt, 
jaki mo"rna wywie"s"c dla tej sygnatury, spe"lnia $\phi$.
Tak wi"ec, w sygnaturze $insert(P, T)$ aksjomat $P$ wymusza zr"ownanie typ"ow \|t| i~\|u|.

\subsection{\secanaleml}
\label{sec:analeml}

Semantyka Statyczna EML jest niedeterministyczna.
W dodatku niekt"ore z przes"lanek wyst"epuj"acych w regu"lach
wyra"raj"a w"lasno"sci postaci
"`dla ka"r\-de\-go wywodu takiego "re $\phi$ zachodzi,
$\psi$ musi zachodzi"c'' --- dobrymi przyk"ladami
s"a wymagania g"l"owno"sci sygnatur pojawiaj"ace si"e w Semantyce Statycznej Mo\-du\-"l"ow.

Z powodu niedeterminizmu i kwantyfikacji po niesko"nczonych zbiorach,
niemo"rliwa jest bezpo"srednia implementacja Semantyki Statycznej EML.
Na\-le\-"ry przet"lumaczy"c j"a do bardziej "`denotacyjnej'' postaci
(niekt"orzy wol"a nazywa"c t"e posta"c "`deterministyczn"a'').
Dopiero wtedy mo"re ona zosta"c niemal dos"lownie
zaimplementowana w j"ezyku programowania, na przyk"lad w~SML.

Takie przeformu"lowanie jest nietrywialnym zadaniem.
Na szcz"e"scie, po\-dob\-ny problem dotycz"acy SML
zosta"l rozwi"azany ju"r jaki"s czas temu.
Algorytm Damasa i Milnera~\cite{DM82} jest dobrze znanym szkieletem
narz"edzia do sprawdzania poprawno"sci typowej program"ow J"adra SML.
Za"s dow"od twierdzenia o sygnaturach g"l"ownych,
z rozdzia"lu~A.2 klasycznej pozycji~\cite{MT91},
jest przewodnikiem w zmaganiach z j"ezykiem Modu"l"ow SML.
W ko"ncu, sam ML Kit jest bardzo klarown"a i skromn"a
reformulacj"a i implementacj"a Semantyki Statycznej SML,
opart"a na wymienionych powy"rej ideach.

Moja praca nad przeformu"lowaniem specyficznych dla EML
fragment"ow Semantyki Statycznej czasami sprowadza"la si"e 
do trywialnych rozszerze"n na przyk"lad algorytmu Damasa i Milnera,
gdzie jedynym k"lopotem by"lo zmaganie z maszyneri"a systemu ML Kit. 
Tak by"lo w przypadku prac nad kwantyfikatorami i pozosta"lymi konstrukcjami specyfikacyjnymi
oraz w przypadku analizy innych ni"r podstawowe postaci aksjomat"ow.

Czasami jednak rozszerzenie wymaga"lo dowodu poprawno"sci,
albo bardzo dobrej znajomo"sci budowy systemu.
By"lo to potrzebne na przyk"lad przy okazji prac nad semantyk"a funktor"ow EML,
oraz nad ustaleniem poprawno"sci rozwi"aza"n dotycz"acych 
rozwik"lywania prze"ladowania i zbior"ow jawnych zmiennych typowych.

Jeszcze kiedy indziej trzeba by"lo znale"z"c zupe"lnie nowe podej"scie,
udowodni"c jego poprawno"s"c i zmie"sci"c w ramach systemu ML Kit.
Niekt"ore z takich przedsi"ewzi"e"c doprowadzi"ly do propozycji zmian w samej definicji EML.
Tak by"lo w przypadku prac nad zasi"egiem zmiennych typowych, 
nad warunkami poprawno"sci aksjomat"ow, czy nad reprezentacj"a "slad"ow.

Jednak zdecydowanie najwi"eksze trudno"sci sprawia"ly
elaboracja aksjomat"ow w sygnaturach oraz zbieranie "slad"ow analizy statycznej.
Te w"la"snie dwa zagadnienia zostan"a om"owione 
ze szcze\-g"ol\-n"a dok"ladno"sci"a w nast"epnych rozdzia"lach.
Opis innych nietrywialnych problem"ow, 
w szczeg"olno"sci tych wymienionych powy"rej, znajduje si"e w pracy~\cite{JKS96}.
Natomiast wyczerpuj"acy obraz implementacji analizy statycznej EML, w ramach systemu EML~Kit, 
mo"rna uzyska"c jedynie poprzez lektur"e jego kodu "zr"od"lowego, wraz z definicj"a Extended ML.

%Wydaje si"e, "re znajomo"s"c definicji Extended ML%~\cite{bib:KST94} 
%jest niezb"edna do pe"lnego zrozumienia tej pracy.
%Przydatne by"loby r"ownie"r obycie z algorytmem Damasa i Milnera%~\cite{DM82} 
%,opisanym w bardziej eleganckim uj"eciu w pracy~\cite{Ler92}.

%%% Local Variables: 
%%% mode: latex
%%% TeX-master: "eml-type"
%%% End: 
