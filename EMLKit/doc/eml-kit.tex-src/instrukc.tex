\documentstyle[12pt,plfonts]{article}

\textheight=220mm
\textwidth=170mm
\topmargin=-15mm
\oddsidemargin=0mm
\pagestyle{empty}
\parindent=20pt
\parskip=20pt

\def\leadersfill{\leaders\hbox to 4pt{\hss.\hss}\hfill}

\begin{document}
\thispagestyle{empty}

\begin{center}
{\bf Instrukcja przygotowania tekstu zgodnie z formatem\\
 Raport"ow Instytutu Informatyki}
\end{center}

\bigskip

Raporty IInf maj"a ustalony format LateXowy. Style do samodzielnego
przygotowania tekstu mo"zna otrzyma"c od p.~Haliny "Swiderek p.~4050.
Niestety, przesy"lanie tych zbior"ow poczt"a da"lo kiepskie efekty.
Nie zdarzy"la si"e ``czysta'' transmisja, poczta zawsze dostawi"la jakie"s
swoje znaczki (np. !!!) i~wszystko przestawa"lo chodzi"c.

Zestaw obejmuje dwa zbiory: rin11.sty i rinf.sty. Musz"a one zosta"c
wgrane do katalogu, w~kt"orym przygotowujemy dokument. 
Korzystaj"ac ze zbioru exampl.tex nale"zy zast"api"c po\-cz"a\-tek swojego
dokumentu komendami zawartymi w~exampl.tex.

Po wykonaniu tych czynno"sci przetwarzamy dokument LateXem, poprawiamy
ewentualnie powsta"le ma"lo eleganckie fragmenty (u"lo"zenie tabel,
rysunk"ow, podzia"l na strony) i~drukujemy {\bf jednostronnie}.

Tak przygotowany tekst {\it camera ready\/} dostarczamy p.~H.~"Swiderek.

\begin{flushright}
 Joanna Mickiewicz
\end{flushright}
\end{document}



