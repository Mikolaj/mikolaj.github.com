\chapter{
Wyra"renia
}

Program w j"ezyku funkcyjnym to wyra"renie, 
a wykonanie programu to obliczenie warto"sci wyra"renia. Przyk"lady wyra"re"n:
\begin{verbatim}
    3;   (* Comment: semicolon ";" tells the interpreter *)
    (1 + 2) * 1;   (* to evaluate expressions. *)
    ("Hi! I'm the first element of this tuple", 3, false);
    if ((2.3 > 5.4) orelse true) then "yes" else "no";
    (fn n => 2 * n) 3;
    fn x => x;
\end{verbatim}
 
Wielkie i skomplikowane wyra"renia zapisuje si"e wygodniej 
u"rywaj"ac konstrukcji \|let...in...end| oraz deklaracji:
\begin{verbatim}
    let 
       val text = "I'm a string"
    in
       text^". Yes, "^text^". Sure, "^text^"!"
    end;
    val twice = (fn n => 2 * n);
    twice;
    twice (twice 3);
    fun twice' n = 2 * n;
    twice 3 = twice' 3 andalso twice' 4 = twice 4;
\end{verbatim}

Przyk"lad z \|pi| pokazuje pewn"a istotn"a ceche deklaracji.
Jaka jest warto"s"c wyra"renia \|area 5.0|
z ostatniej linii tego przyk"ladu?
\begin{verbatim}
    val pi = 3.14;
    fun area r = pi * r * r;
    area 5.0;
    val pi = 1.0;
    area 5.0;
\end{verbatim}

% W programach funkcyjnych pojawiaj"a si"e r"ownie"r deklaracje rekurencyjne:
Pot"e"rnym narz"edziem programowania funkcjnego jest rekurencja:
\begin{verbatim}
    fun fact n = if n <= 1 then 1 else n * (fact (n - 1)); 
    fun fib n = if n <= 1 then 1 else (fib (n - 1)) + (fib (n - 2));
    fact (fib 4);
    fib;
\end{verbatim}
% \ldots\ i wielkie mn"ostwo innych cudowno"sci, a wszystko po to, 
% by uda"lo si"e zapisa"c to wyra"renie, na kt"orego warto"s"c 
% od pocz"atku pisania programu dybiemy.

\section*{Zadanie}
\begin{exercises}

\item
Oto definicja funkcji \|fib'|:
\begin{verbatim}
fun fib' n = 
    let
        fun fib_pair n = 
            if n <= 1 then (1, 1) 
                      else let
                               val (j, k) = fib_pair (n - 1)
                           in
                               (j + k, j)
                           end
    in
        #1 (fib_pair n)
    end
\end{verbatim}
(\|#1| oznacza wzi"ecie pierwszego elementu tupla).
 
Udowodnij przy pomocy indukcji, "re dla ka"rdego n
$$\f{fib} n = \f{fib'} n.$$

Koszt czasowy $\f{fib} n$ jest wyk"ladniczy wzgl"edem $n$. 
Jaki jest koszt $\f{fib'} n$?

\end{exercises}


%%% Local Variables: 
%%% mode: latex
%%% TeX-master: "TajnySkryptSML"
%%% End: 
