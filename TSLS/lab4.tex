\chapter{
Definiowanie typ"ow
}

Mo"rna definiowa"c typy:
\begin{verbatim}
    datatype friend = PAT | SUSIE | JOE   
    datatype gift = BOTTLES_OF_BEER of int | BOOK
\end{verbatim}
a przy pomocy pattern matchingu na tych typach tworzy"c funkcje:
\begin{verbatim}
    fun give PAT = BOTTLES_OF_BEER(10)
      | give SUSIE = BOOK
      | give JOE = BOTTLES_OF_BEER(3)
    fun like (BOTTLES_OF_BEER(n)) = n > 7
      | like BOOK = true
\end{verbatim}
Typy mog"a by"c rekurencyjne:
\begin{verbatim}
    datatype int_list = INT_NIL | INT_CONS of int * int_list
    fun sum INT_NIL = 0
      | sum (INT_CONS(i, rest)) = i + sum rest 
\end{verbatim}
wzajemnie rekurencyjne:
\begin{verbatim}
    datatype friend_rec = PAT | SUSIE | ANYONE_WHO_GIVES_ME of gift_rec
    and gift_rec = BOTTLES_OF_BEER of int | PHOTO of friend_rec
    fun love PAT = false
      | love SUSIE = true
      | love (ANYONE_WHO_GIVES_ME(a_gift)) = like a_gift
    and like (BOTTLES_OF_BEER(n)) = n > 7
      | like (PHOTO(my_friend)) = love my_friend
\end{verbatim}
polimorficzne :
\begin{verbatim}
    datatype 'a option = SOME of 'a | NONE
    val int_some_list = [SOME(5), NONE, SOME(10), SOME(3), NONE]
    val bool_some_list = [SOME(true), SOME(true), NONE, SOME(true)]
    fun divide (NONE, _) = NONE
      | divide (_, NONE) = NONE
      | divide (SOME(x), SOME(y)) = if y = 0.0 then NONE else SOME(x / y)
\end{verbatim}

\begin{verbatim}
    fun pair_some (SOME(x), SOME(y)) = SOME(x, y)
      | pair_some _ = NONE
    fun divide2 (r, s) = 
        case (pair_some (r, s)) 
          of NONE => NONE
           | (SOME(x, y)) => if y = 0.0 then NONE else SOME(x / y)
\end{verbatim}

Znany nam wbudowany typ \|list| 
jest rekurencyjny i polimorficzny
i ma nast"epuj"ac"a definicj"e:
\begin{verbatim}
    infixr 5 ::
    datatype 'a list = nil | :: of ('a * 'a list)
\end{verbatim}
A drzewo mo"rna zdefiniowa"c na przyk"lad tak:
\begin{verbatim}
    datatype 'a tree = EMPTY | NODE of 'a tree * 'a * 'a tree
\end{verbatim}


\section*{Zadania}
\begin{exercises}

\item
Napisz wielkie mn"ostwo funkcji na drzewach.
Mo"resz napisa"c odpowiedniki wszystkich znanych Ci funkcji na listach.
Mo"resz znale"z"c jakie"s funkcje specyficzne dla drzew, 
na przyk"lad: obchodz"ace drzewo, zamieniaj"ace prawe poddrzewa z lewymi, etc.
Mo"resz napisa"c funkcje z drzew w listy i z list w drzewa. 

\item
Zaimplementuj sortowanie drzewowe (treesort).

Wskaz"owka: W uporz"adkowanym drzewie binarnym liczba wyst"epuj"aca w w"e"zle
jest wi"eksza od wszystkich w lewym poddrzewie i mniejsza od wszyskich w prawym.
Nale"ry zdefiniowa"c dwie g"l"owne funkcje. Jedn"a, kt"ora dostaje list"e liczb naturalnych, 
za"s oddaje uporz"adkowane drzewo binarne zawieraj"ace elementy tej listy. 
Drug"a, kt"ora dostaje uporz"adkowane drzewo binarne, a oddaje
posortowan"a list"e jego element"ow. Treesort to z"lo"renie tych funkcji. 

%\item
%Zaimplementuj sortowanie stogowe (heapsort).

%Wskaz"owka: st"og (heap) to drzewo, w kt"orym ka"rdy element jest mniejszy-r"owny 
%od ka"rdego elementu wyst"epuj"acego pod nim. Nale"ry zdefiniowa"c
%dwie g"l"owne funkcje. Jedn"a, kt"ora dostaje list"e liczb naturalnych, za"s oddaje
%heap zawieraj"acy elementy tej listy. Drug"a, kt"ora dostaje heap, a oddaje
%posortowan"a list"e jego element"ow. Do zrobienia tej drugiej zapewne przyda si"e
%funkcja, kt"ora z dw"och heap"ow robi jeden.

\item
To jest bardzo "ladna (i odrobink"e nieefektywna) definicja liczb naturalnych:
\begin{verbatim} 
    datatype nat = ZERO | SUCC of nat
\end{verbatim}
\|SUCC| od successor - nast"epnik.
Napisz kilka podstawowych funkcji i relacji (czyli funkcji typu \|nat->bool|)
na tych liczbach naturalnych.

\item
Oto drzewo sk"ladni abstrakcyjnej (gramatyka) pewnego j"ezyka wyra"re"n
\begin{verbatim} 
    datatype expression =
        NUM of int
      | TIMES of expression * expression
      | PLUS of expression * expression
\end{verbatim}
Napisz funkcj"e wyliczaj"ac"a warto"s"c wyra"renia.
Napisz wersj"e $LL1$ po\-wy"r\-szej gramatyki.
Napisz funkcj"e przekszta"lcaj"ac"a wyra"renie
w gramatyce $LL1$ w wyra"renie w podanej wy"rej gramatyce.
Napisz parser (u"rywaj"ac gramatyki $LL1$) i lekser.

Wskaz"owka:
\begin{verbatim}
<1st_atom>   -> <numeral>
<2nd_atom>   -> <1st_atom> <2nd_atom'>
<2nd_atom'>  -> epsilon | * <1st_atom> <2nd_atom'>
<3rd_atom>   -> <2nd_atom> <3rd_atom'>
<3rd_atom'>  -> epsilon | + <2nd_atom> <3rd_atom'>
<expression> -> <3rd_atom>
\end{verbatim}

\item
Zdefiniuj typ drzew, w kt"orych ka"rdy wierzcho"lek mo"re mie"c
dowolnie wielu syn"ow. Napisz troch"e funkcji na tych drzewach.

\end{exercises}


%%% Local Variables: 
%%% mode: latex
%%% TeX-master: "TajnySkryptSML"
%%% End: 
