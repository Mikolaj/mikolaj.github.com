\chapter{
Funkcje jeszcze wy"rszych rz"ed"ow 
}

Funkcje wy"rszych rz"ed"ow spotyka si"e wsz"edzie. We"zmy cho"cby
aplikacj"e funkcji do argumentu. To"c to tw"or wy"rszego rz"edu! Mo"remy go nazwa"c: 
\begin{verbatim}
    fun apply (f, a) = f a
\end{verbatim}
i od tej pory zawsze pisa"c \|apply (g, b)| tam gdzie dot"ad pisali"smy \|g b  :-]|.

Operacja sk"ladania dw"och funkcji jest oczywi"scie funkcj"a wy"rszego rz"edu. 
Pewna jej wersja jest nazwana i zdefiniowana w standardowej bazie SML-a:
\begin{verbatim}
    infix 3 o
    fun (f o g) x = f (g x)
\end{verbatim}
i czasem si"e przydaje, np.:
\begin{verbatim}
    val treesort = bst2list o list2bst
\end{verbatim}
Nawet pospolite operacje arytmetyczne mo"rna zapisa"c jako funkcje wy"rszego rz"edu:
\begin{verbatim}
    fun divide_by x y = y / x;
    divide_by 2.0 4.0;
    val divide_by_two = divide_by 2.0;
    divide_by_two 10000000.0;
    divide_by_two 342423423452345.0;
    map (divide_by 3.0) [333.0, 12.0, 55.0, 5.0];
\end{verbatim}

Definiuj"ac \|divide_by| na podstawie funkcji ~\|/|~ dokonali"smy pewnych
my\-"slo\-wych operacji wy"rszego rz"edu (jak"reby inaczej). 
Te operacje mo"rna zapisa"c w SML-u:
\begin{verbatim}
    fun curry f a b = f (a, b)
    fun C f x y = f y x
\end{verbatim}
Definicj"e \|divide_by| mo"rna teraz przedstawi"c tak:
\begin{verbatim}
    val divide_by = C (curry (op /))
\end{verbatim}

Tak na marginesie: dla dowolnego \|f : 'a * 'b -> 'c| istnieje dok"ladnie
jedna funkcja \|g : 'a -> 'b -> 'c| taka, "re dla ka"rdego \|(x, y) : 'a * 'b|
\|f (x, y)| $=$ \|g x y|. Tym \|g| jest \|curry f|. Sk"ad wynika na przyk"lad, "re ka"rd"a 
operacj"e arytmetyczn"a mo"rna przerobi"c na funkcj"e wy"rszego rz"edu tylko na dwa sposoby.

Jeszcze jedno --- przy pomocy lambda notacji
funkcje wy"rszych rz"ed"ow zapisuje si"e zagnie"rd"raj"ac \|fn|, np.:
\begin{verbatim}
    fun curry f = fn a => fn b => f (a, b)
    val C = fn f => fn x => fn y => f y x
\end{verbatim}


\section*{Zadania}
\begin{exercises}

\item%[Zadanie 6.1]
Zdefiniuj funkcje odwrotne do \|curry| (b"edzie si"e nazywa"la \|uncurry|)
i \|C| (nazw"e wymy"sl sama (sam)).
Napisz funkcj"e \|repeat|, kt"ora dostawszy pare \|(f, n)| daje
funkcj"e r"own"a z"lo"reniu \|n| funkcji \|f|. Czyli:
$$ \f{repeat}(f,n) = f^n $$

\item%[Zadanie 6.2]
Dla dowolnych dw"och typ"ow \|'a| i \|'b| istnieje typ \|'a * 'b|, ich produkt.
W naturalny spos"ob zwi"azane s"a z tym typem pewne funkcje ---
projekcje:
\begin{verbatim}
    fun pi1 (x, y) = x
    fun pi2 (x, y) = y
\end{verbatim}
i funkcja \|pair| typu \|('c -> 'a) * ('c -> 'b) -> ('c -> 'a * 'b)|,
kt"ora dostawszy funkcje \|f :'c -> 'a| oraz \|g : 'c -> 'b| daje funkcj"e\\
 \|h : 'c -> 'a * 'b|, o tej w"lasno"sci, "re
\|pi1 o h = f| i \|pi2 o h = g|. 
(Taka funkcja \|h| jest dla ka"rdego \|(f, g)| tylko jedna.)

Napisz funkcj"e \|pair|.

\item%[Zadanie 6.3]
Zdefiniuj typ \|('a, 'b) direct_sum|, kt"ory jest sum"a roz"l"aczn"a ty"ow \|'a| i \|'b|.
Ten typ b"edzie w pewnym sensie dualny do typu \|'a * 'b|.
Zdefiniuj w"lo"renia (injekcje):
\begin{verbatim}
    inleft  : 'a -> ('a, 'b) direct_sum 
    inright : 'b -> ('a, 'b) direct_sum
\end{verbatim} 
i funkcj"e \|cases| analogiczn"a (tak naprawd"e dualn"a) do funkcji \|pair| dla produktu.

Wskaz"owka 1:
Funkcja \|cases| jest typu\\
 \|('a -> 'c) * ('b -> 'c) -> ('a, 'b) direct_sum -> 'c|.

Wskaz"owka 2:
Funkcja \|cases| dostawszy funkcje \| f : 'a -> 'c | oraz\\
\| g : 'b -> 'c |
daje funkcj"e \| h : ('a, 'b) direct_sum -> 'c |, o tej w"lasno"sci, "re
\| h o inleft = f | i \| h o inright = g |. 
(Taka funkcja \|h| jest dla ka"rdego \|(f, g)| tylko jedna.)

\pagebreak

\item%[Zadanie 6.4]
Napisz funkcj"e 
\begin{verbatim}
    exists : ('a -> bool) -> 'a list -> bool
\end{verbatim} 
kt"ora bierze predykat
\|p : ('a -> bool)| i daje funkcj"e, kt"ora bierze list"e \|l| i oddaje \|true|
gdy kt"ory"s z element"ow listy spe"lnia predykat \|p|, \|false| w przeciwnym przypadku.
Napisz funkcj"e 
\begin{verbatim}
    non : ('a -> bool) -> ('a -> bool)
\end{verbatim} 
kt"ora neguje predykat.
Napisz funcj"e \|forall|, kt"ora tym si"e r"o"rni od \|exists|, "re w odpowiednim
miejscu nie kt"ory"s, ale wszystkie elementy listy musz"a spe"lnia"c predykat.
Nim zaczniesz pisa"c \|forall| podobnie jak \|exists|, 
pomy"sl czy nie mo"rnaby go napisa"c "smieszniej 
(nie, nie chodzi mi o to "reby u"ryc \|foldr|).

Spoiler:
W definicji \|forall| korzystaj wy"lacznie z \|exists|, \|non|, i \|o|.

\item%[Zadanie 6.5]
%\frenchspacing
Zapisz insertionsort lub inny algorytm sortowania 
(mo"ze mergesort, je"sli jeszcze go nie napisa"le"s lub nie napisa"la"s) w takiej
postaci, aby jako pierwszy argument bra"l funkcj"e \|le : 't * 't -> bool| wyznaczj"ac"a
pewien porz"adek i sortowa"l list"e \|l : 't list| wed"lug tego porz"adku.
Napisz \|le_int| i \|le_string|. 
Wypr"obuj \|insertionsort le_int|.
Napisz te"r jaki"s 
\begin{verbatim}
    le_int_to_int : (int -> int) * (int -> int) -> bool
\end{verbatim}
Wypr"obuj \|insertionsort le_int_to_int|.

\end{exercises}


%%% Local Variables: 
%%% mode: latex
%%% TeX-master: "TajnySkryptSML"
%%% End: 
