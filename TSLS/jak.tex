\chapter{
Jak pracowa"c z kompilatorem SML-a}

\begin{itemize}

\item Interaktywnie:
\begin{enumerate}
\item uruchomi"c SML (\|mosml|, \|new-sml|, \|sml-0.93|, \|sml|)
\item po zach"ecie, kt"or"a na og"o"l jest my"slnik, napisa"c program
\item zako"nczy"c pisanie "srednikiem i nacisn"a"c \|RET|
\end{enumerate}

\item Wsadowo: 

\begin{enumerate}
\item napisa"c program jakim"s edytorem
\item zapisa"c go w pliku \|smth.sml|
\item uruchomi"c SML i napisac \|use "smth.sml";|
\end{enumerate}

\item Przy pomocy EMACS-a i \|sml-mode|:

\begin{enumerate}
\item uruchomi"c EMACS-a
\item \|C-x C-f smth.sml RET| (wa"rne, "reby ko"nc"owk"a nazwy by"lo \|.sml|)
\item napisa"c b"l"edny program (\|Tab| automagicznie formatuje)
\item \|C-c C-l| i b"edzie wiadomo co dalej
\item Zrobi"c \|C-c `| (ten ostatni znaczek to back-quote, 
      wyst"epuje na klawiaturze na og"o"l pod tyld"a), 
      wtedy on ustawi kursor tam gdzie w pliku jest b"l"ad 
\item Korzysta"c z~innych dobrodziejstw \|sml-mode|,
      np.\ \|sml-mode-info|, czyli podr"ecznika do \|sml-mode|
\end{enumerate}

\end{itemize}

%%% Local Variables: 
%%% mode: latex
%%% TeX-master: "TajnySkryptSML"
%%% End: 
