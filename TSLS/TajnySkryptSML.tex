\documentclass[a4paper]{report}
\usepackage[polish]{babel}
%\documentstyle[a4,12pt,polish]{book}

\flushbottom

\newcounter{ex}[chapter]
\def\Item[#1]{\item}
\newenvironment{exercises}
{\begin{list}{{\bf Zadanie \thechapter.\theex}}{\usecounter{ex}}}
{\end{list}}
\let\zadania\exercises
\let\endzadania\endexercises
\def\f#1{\ifmmode\mathord{\mbox{\em #1}\;}\else{\em #1\ }\fi}
%\def\f#1{\ifmmode\mathord{\mbox{\|#1|}\;}\else{\|#1|\ }\fi}
\def\|{\verb|}

%%%%%%%%%%%%%%%%%%%
%
%\includeonly{lab9i10}
%%%
%\sfvariables
%%%%%%%%%%%%%%%%%%%

\begin{document}

\author{Marcin Benke\thanks{
   Instytut Informatyki, 
% Wydzia"l Matematyki, Informatyki i Mechaniki, 
   Uniwersytet Warszawski.}
\\{\tt benke@mimuw.edu.pl}\\\emph{redakcja i rozdzia"l~\ref{MBchapter}} 
\and
Grzegorz Grudzi"nski$^2$\\{\tt gsg@mimuw.edu.pl}\\\emph{rozdzia"l~\ref{GGchapter}}
\and
Miko"laj Konarski$^2$\\{\tt mikon@mimuw.edu.pl}\\\emph{pozosta"le rozdzia"ly}}

%\footnotemark[\value{footnote}]

\title{Tajny Skrypt Lab SML\thanks{Nazwa jest historyczna.}\\v 0.33}

%\special{!userdict begin /bop-hook{gsave 200 30 translate 60 rotate
%/Times-Roman findfont 150 scalefont setfont 90 90 moveto 0.8 setgray
%(TAJNE) show grestore}def end}

\maketitle

%\begin{titlepage}
%\hbox{}
%\clearpage
%\end{titlepage}
\tableofcontents

\chapter*{Ostrze"renie}

\begin{verbatim}
#include <std/disclaimer.h> ;-)
\end{verbatim}

Ten dokument, zgodnie ze swym tytu"lem, nie tylko {\bf nie jest
oficjalnym skryptem}, ale co wi"ecej --- {\bf jest tajny}. 
%Oznacza to "re
%pod gro"zb"a pewnych nieprzyjemno"sci 
%(np.\ ekskomuniki)\footnote{{\em
%ekskomunika} --- odci"ecie "l"aczno"sci, np.\ Internetu}
%{\bf  niewskazane} jest jego udost"epnianie osobom nie zwiazanym z
%pracowni"a SML.  

Ponadto nie s"a to notatki do wyk"ladu, tylko do {\bf
pracowni}. Wiedza podawana jest tu wyrywkowo i skr"otowo,
%i dla przypomnienia,
a nacisk po"lo"rony jast na {\bf zadania}.
%
%Ten dokument powsta"l bez wiedzy i zgody w"ladz Wydzia"lu, a mo"re
%nawet przy ich zdecydowanym sprzeciwie.
%[^Zasz"lo"s"c historyczna.]

\bigskip
\rightline{\em Tajne Kolegium Redakcyjne}
%%% Local Variables: 
%%% mode: latex
%%% TeX-master: "TajnySkryptSML"
%%% End: 

\chapter{
Wyra"renia
}

Program w j"ezyku funkcyjnym to wyra"renie, 
a wykonanie programu to obliczenie warto"sci wyra"renia. Przyk"lady wyra"re"n:
\begin{verbatim}
    3;   (* Comment: semicolon ";" tells the interpreter *)
    (1 + 2) * 1;   (* to evaluate expressions. *)
    ("Hi! I'm the first element of this tuple", 3, false);
    if ((2.3 > 5.4) orelse true) then "yes" else "no";
    (fn n => 2 * n) 3;
    fn x => x;
\end{verbatim}
 
Wielkie i skomplikowane wyra"renia zapisuje si"e wygodniej 
u"rywaj"ac konstrukcji \|let...in...end| oraz deklaracji:
\begin{verbatim}
    let 
       val text = "I'm a string"
    in
       text^". Yes, "^text^". Sure, "^text^"!"
    end;
    val twice = (fn n => 2 * n);
    twice;
    twice (twice 3);
    fun twice' n = 2 * n;
    twice 3 = twice' 3 andalso twice' 4 = twice 4;
\end{verbatim}

Przyk"lad z \|pi| pokazuje pewn"a istotn"a ceche deklaracji.
Jaka jest warto"s"c wyra"renia \|area 5.0|
z ostatniej linii tego przyk"ladu?
\begin{verbatim}
    val pi = 3.14;
    fun area r = pi * r * r;
    area 5.0;
    val pi = 1.0;
    area 5.0;
\end{verbatim}

% W programach funkcyjnych pojawiaj"a si"e r"ownie"r deklaracje rekurencyjne:
Pot"e"rnym narz"edziem programowania funkcjnego jest rekurencja:
\begin{verbatim}
    fun fact n = if n <= 1 then 1 else n * (fact (n - 1)); 
    fun fib n = if n <= 1 then 1 else (fib (n - 1)) + (fib (n - 2));
    fact (fib 4);
    fib;
\end{verbatim}
% \ldots\ i wielkie mn"ostwo innych cudowno"sci, a wszystko po to, 
% by uda"lo si"e zapisa"c to wyra"renie, na kt"orego warto"s"c 
% od pocz"atku pisania programu dybiemy.

\section*{Zadanie}
\begin{exercises}

\item
Oto definicja funkcji \|fib'|:
\begin{verbatim}
fun fib' n = 
    let
        fun fib_pair n = 
            if n <= 1 then (1, 1) 
                      else let
                               val (j, k) = fib_pair (n - 1)
                           in
                               (j + k, j)
                           end
    in
        #1 (fib_pair n)
    end
\end{verbatim}
(\|#1| oznacza wzi"ecie pierwszego elementu tupla).
 
Udowodnij przy pomocy indukcji, "re dla ka"rdego n
$$\f{fib} n = \f{fib'} n.$$

Koszt czasowy $\f{fib} n$ jest wyk"ladniczy wzgl"edem $n$. 
Jaki jest koszt $\f{fib'} n$?

\end{exercises}


%%% Local Variables: 
%%% mode: latex
%%% TeX-master: "TajnySkryptSML"
%%% End: 
 
\chapter{
Listy \& stuff
}

Poznali"smy tuple: 
\begin{verbatim}
    (2, "errr", 3.14);
\end{verbatim}
Listy to co"s zupe"lnie innego:
\begin{verbatim}
    [2, 55, 15346];
    ["This", "is", "a", "list", ", yeah."];
    [];
\end{verbatim}

Te nawiasy kwadratowe to skr"otowy spos"ob zapisu list.
Tak naprawd"e listy konstruuje si"e w ten spos"ob:
\begin{verbatim}
    2::(55::(15346::nil));
\end{verbatim}
lub bez niepotrzebnych nawias"ow:
\begin{verbatim}
    2::55::15346::nil;
    "This"::"is"::"a"::"list"::", yeah."::nil;
    nil;
\end{verbatim}

M"owi"ac "sci"sle: infiksowy konstruktor \|::| dostaje jako argumenty
warto"s"c~$x$ i list"e~$l$, kt"orej elementy s"a tego samego typu co~$x$,
i daje w~wyniku $l$ z doklejonym na pocz"atku elementem $x$.
Na przyk"lad (u"rywaj"ac mieszanej notacji) \|1::[2, 3, 4]| 
reprezentuje t"e sam"a list"e co  \|[1, 2, 3, 4]|

Kilka przyk"lad"ow:
\begin{verbatim}
    val mylist = [1, 2, 3];
    fun null l = (l = []);
    null mylist;
    null [];
    fun interval (n, m) = if n > m then [] else n::interval (n+1, m);
    interval (~31, 72);
\end{verbatim}

Jest troche wbudowanych funkcji na listach,
w szczeg"olno"sci
\|hd| oddaj"aca pierwszy element listy i~\|tl|, oddaj"aca list"e bez pierwszego elementu.
\begin{verbatim}
    val head = hd mylist
    val tail = tl mylist
    val what = head :: tail
\end{verbatim}

Funkcje na listach mo"rna definiowac za pomoca pattern matchingu
\begin{verbatim}
    fun null [] = true
      | null (head::tail) = false
\end{verbatim}
a tu b"edzie potrzebny wyj"atek (exception):
\begin{verbatim}
    exception Empty
    fun hd [] = raise Empty
      | hd (head::tail) = head
\end{verbatim}

Dalej ju"r "latwo \|:-)|


\section*{Zadania}
\begin{exercises}

\item%[Zadanie 2.1] 
Przy pomocy pattern matchingu napisz funkcj"e \|tl|.

\item%[Zadanie 2.2]
U"rywaj"ac funkcji \|null|, \|hd|, \|tl|,
napisz funkcj"e \|sum|, kt"ora dodaje do kupy
wszystkie elementy danej listy liczb ca"lkowitych.
Zr"ob to samo przy pomocy pattern matchingu.

\item%[Zadanie 2.3]
U"rywaj"ac funkcji \|null|, \|hd|, \|tl|,
napisz funkcj"e \|length|, kt"ora oblicza d"lugo"s"c listy.
Zr"ob to samo przy pomocy pattern matchingu.

\item%[Zadanie 2.4]
U"rywaj"ac funkcji \|null|, \|hd|, \|tl|,
napisz funkcj"e \|last|, kt"ora daje w wyniku ostatni element listy.
Zr"ob to samo przy pomocy pattern matchingu.

\item%[Zadanie 2.5]
Napisz funkcj"e \|append|, kt"ora "l"aczy dwie listy.
Na przyk"lad:
 $$\f{append} ([1, 2], [3, 4, 6]) = [1, 2, 3, 4, 6]$$

\item%[Zadanie 2.6]
Napisz funkcje \|take| i \|drop|, z parametrem \|(l,n)|,
kt"ore odpowiednio bior"a lub odrzucaj"a 
pierwsze \|n| element"ow listy \|l|.
Upewnij si"e, "re dla ka"rdego $(l, n)$ 
$$
\f{append} (\f{take} (l, n), \f{drop} (l, n)) = l
$$

%\item%[Zadania 2.8a]
%Czy wiesz dlaczego w poprzednim zadaniu napisane jest ``z parametrem \|(l,n)|'', a
%nie ``z parametrami \|(l,n)|''?

\item%[Zadanie 2.7]
Napisz, korzystaj"ac z funkcji \|drop| i~\|hd|,
funkcj"e \|nth|, kt"ora daje $n$-ty element listy.
Czy funkcja dzia"la"laby szybciej, gdyby skorzysta"c z~\|take| i \|last|? 

\item%[Zadanie 2.8]
Napisz funkcj"e \|flatten|, kt"ora dostaje list"e list, 
a daje w wyniku list"e wszystkich element"ow tych list
(nie martw si"e powt"orzeniami czy kolejno"sci"a).

\item%[Zadanie 2.9]
Napisz \|quicksort|.

\item%[Zadanie 2.10]
Przy pomocy list napisz funkcje realizuj"ace arytmetyk"e na DU\-"RYCH
liczbach ca"lkowitych. U"ryj ich do policzenia $500!$ albo lepiej
$1000!$. Jaka jest z"lo"rono"s"c liczenia $n!$ w ten spos"ob?

\item%[Zadanie 2.11]
Napisz to samo w C lub Pascalu. Por"ownaj czas zu"ryty na pisanie i
wykonanie programu.

\end{exercises}

Poza tym mo"rna zrobi"c \|mergesort| albo funkcje obliczaj"ace
wszystkie permutacje, kombinacje, etc. danej listy 
albo par"e funkcji: \|zip| i~\|split|; \|zip| dostaje dwie listy
liczb ca"lkowitych a oddaje list"e par: para pierwszych element"ow
obu list, potem drugich, itd, \|split| jest funkcj"a (prawie) odwrotn"a.

Wszystkie napisane programy warto przetestowa"c ---
jest to "latwe i przyjemne, bo u"rywamy interpretera. 

{\bf Zach"ecamy do wszelakich eksperyment"ow!}

%%% Local Variables: 
%%% mode: latex
%%% TeX-master: "TajnySkryptSML"
%%% End: 

\label{lab:list}
\chapter{
Rekursja ogonowa, akumulatory i r"o"rne takie
}

Cz"esto funkcj"e mo"rna zdefiniowa"c
na kilka istotnie ro"rnych sposob"ow.
Sztandarowym przyk"ladem jest funkcja daj"aca $n$--t"a
liczb"e Fibonacciego, w"sr"od kt"orej najprostszych
definicji istniej"a drastyczne ro"rnice z"lo"rono"sci czasowej.
Du"r"a rodzin"e innych przyk"lad"ow stanowi"a funkcje przy kt"orych
definiowaniu mo"rna ale nie trzeba u"rywa"c \emph{akumulatora}
tzn.\ dodatkowego parametru kt"ory kumuluje sk"ladowe wyniku.

Oto definicja silni bez akumulatora:
\begin{verbatim}
fun fact n = if n <= 1 then 1 else n * fact (n - 1)
\end{verbatim}
i z akumulatorem:
\begin{verbatim}
fun fact_tr n = 
    let
        fun fact_acc (acc, k) = if k <= 0 then acc 
                                else fact_acc (acc * k, k - 1)
    in
        fact_acc (1, n)
    end
\end{verbatim}
przy czym wida"c, "re dla ka"rdego $n$ mamy
 $$\f{fact} n = \f{fact\_tr} n = n!.$$

Jak wygl"ada obliczanie \|fact 4|?
Funkcja \|fact 4| wywo"luje \|fact 3|, zapami"etuj"ac, "re wynik tego wywo"lania
powinna pomno"ry"c przez 4, nim odda go jako sw"oj wynik.
Funkcja \|fact 3| wywo"luje \|fact 2|, zapami"etuj"ac, "re wynik tego wywo"lania
powinna pomno"ry"c przez 3.
Funkcja \|fact 2| wywo"luje \|fact 1|, zapami"etuj"ac, "re wynik tego wywo"lania
powinna pomno"ry"c przez 2. 
Teraz \|fact 1| oddaje 1. Nast"epnie \|fact 2| mno"ry to przez 2 i oddaje 
rezultat tego mno"renia, czyli 2. Potem \|fact 3| mno"ry to przez 3 i~daje 6. 
Potem \|fact 4| mno"ry to przez 4 i oddaje 24.
A wi"ec kolejno"s"c mno"re"n by"la nast"epuj"aca: $((1 * 2) * 3) * 4$.

\pagebreak

Twierdz"e, "re kolejno"s"c mno"re"n przy obliczaniu \|fact_tr 4|
jest odwrotna, czyli $((4 * 3) * 2) * 1$.
Zobaczmy: \|fact_tr 4| wywo"luje \|fact_acc (1, 4)|, kt"ory po wykonaniu
jednego mno"renia na rozruch wywo"luje \|fact_acc (4, 3)|.
Nast"epnie \|fact_acc (4, 3)| mno"ry 4 przez 3 i wywo"luje \|fact_acc (12, 2)|.
Nast"epnie \|fact_acc (12, 2)| mno"ry 12 przez 2 i wywo"luje \|fact_acc (24, 1)|.
Nast"epnie \|fact_acc (24, 1)| mno"ry 24 przez 1 i wywo"luje \|fact_acc (24, 0)|.
\begin{description}
\item[(*)] A teraz \|fact_acc (24, 0)| oddaje 24 
    i wynik ten zwracany jest ju"r bez zmian 
    przez kolejno \|fact_acc (24, 1)|,
    \|fact_acc (12, 2)|, \|fact_acc (4, 3)|, \|fact_acc (1, 4)| i \|fact_tr 4|.
\end{description}

Skoro kolejno"s"c mno"re"n jest odwrotna, \|fact| i~\|fact_tr|
rzeczywi"scie istotnie si"e od siebie r"o"rni"a.
R"o"rnic jest wi"ecej. Na przyk"lad definicja \|fact| jest
niew"atpliwie bardziej czytelna od \|fact_tr|. 
A z drugiej strony w \|fact_tr| wyst"epuje wy"l"acznie
tzw. rekursja ogonowa (tail recursion),
przez co \|fact_tr| wykonuje si"e nieco szybciej
i zu"rywa nieco mniej pami"eci ni"r \|fact|, w kt"orym rekursja
nie jest ogonowa.

Kiedy funkcja zawiera rekursj"e ogonow"a?
Wtedy, kiedy w definicji funkcji wo"lanie rekurencyjne nie jest podwyra"reniem
"radnego innego wyra"renia (z wyj"atkiem drugiej lub trzeciej ga"lezi 
wyra"renia warunkowego).
Je"sli rekursja jest ogonowa, wynik wykonania ka"rdej 
funkcji wo"lanej jest przekazywany bez modyfikacji jako wynik funkcji
wo"laj"acej, tak jak w miejscu oznaczonym (*).

Ka"rdy przywoity kompilator zast"epuje sekwencj"e
powrot"ow tak"a jak (*) jedn"a instrukcj"a skoku,
a tak"re u"rywa jednego rekordu aktywacji dla wszystkich
rekurencyjnych wciele"n funkcji. W ten spos"ob rekurencja jest
automatycznie zast"epowana iteracj"a.
%
% Wida"c, "re w wypadku \|fact| taka optymalizacja nie jest mo"rliwa,
% gdy"r \|fact| w"la"snie w rekordach aktywacji zapami"etuje jakie mno"renia
% ma wykona"c i podczas sekwencji powrot"ow te mno"renia wykonuje.
%
% Oczywi"scie fakt, "re jaka"s definicja funkcji jest ogonowa lub nieogonowa,
% albo u"rywa akumulatora, b"ad"z si"e bez niego obywa, nie wystarcza by j"a oceni"c.
% Zawsze trzeba rozwa"ry"c wszelkie wady i zalety alternatywnych definicji.


\section*{Zadania}
\begin{exercises}

\item%[Zadanie 3.1]
Napisz wersj"e funkcji \|length| z~rekursj"a ogonow"a.

\item%[Zadanie 3.2]
Napisz funkcj"e \|decimal|, kt"ora dostaje list"e cyfr
i odaje liczb"e jaka powstaje z tych cyfr,
je"sli je ustawi"c za sob"a (or something).
np.\ \|decimal [1, 9, 9, 5]| ma da"c 1995.

\item%[Zadanie 3.3]
Napisz dwie wersje funkcji \|reverse| odwracaj"acej list"e.
Jedna z tych wersji ma dzia"la"c w czasie liniowym.

\item%[Zadanie 3.4]
Czy mo"rna napisa"c ogonow"a wersj"e funkcji \|take|?
Je"sli tak, to czy warto jej u"rywa"c?
Odpowied"z uzasadnij pe"lnymi zdaniami.

\item%[Zadanie 3.5]
Napisz wersj"e funkcji \|interval| (z poprzedniego rozdzia"lu)
z rekursj"a ogonow"a. Nie u"rywaj \|reverse| \|;-)|.

\item%[Zadanie 3.6]
Napisz funkcj"e \|real_max|, kt"ora dostaje list"e liczb rzeczywistych
i~oddaje najwi"eksz"a z nich. Napisz dwie istotnie r"o"rne wersje.
Kt"ora wersja bardziej Ci si"e podoba ?

\pagebreak

\item%[Zadanie 3.7]
To inna wersja funkcji \|quicksort|:
\begin{verbatim}
fun quicksort [] = [] : int list
  | quicksort [x] = [x]
  | quicksort (a::rest) = 
    let fun split(left,right,[]) = 
            quicksort left @ (a::quicksort right)
          | split(left,right,x::l) =
            if x < a then split(x::left,right,l)
            else split(left,x::right,l)
    in split([],[],rest)
    end
\end{verbatim}
Por"ownaj j"a z wersj"a podan"a w rozwi"azaniu zadania z~pracowni~\ref{lab:list}.
Czytelno"s"c, ogonowo"s"c, akumulatorowo"s"c, etc.

\end{exercises}

{\bf Jak zwykle eksperymenty surowo wskazane!}

%%% Local Variables: 
%%% mode: latex
%%% TeX-master: "TajnySkryptSML"
%%% End: 

\chapter{
Definiowanie typ"ow
}

Mo"rna definiowa"c typy:
\begin{verbatim}
    datatype friend = PAT | SUSIE | JOE   
    datatype gift = BOTTLES_OF_BEER of int | BOOK
\end{verbatim}
a przy pomocy pattern matchingu na tych typach tworzy"c funkcje:
\begin{verbatim}
    fun give PAT = BOTTLES_OF_BEER(10)
      | give SUSIE = BOOK
      | give JOE = BOTTLES_OF_BEER(3)
    fun like (BOTTLES_OF_BEER(n)) = n > 7
      | like BOOK = true
\end{verbatim}
Typy mog"a by"c rekurencyjne:
\begin{verbatim}
    datatype int_list = INT_NIL | INT_CONS of int * int_list
    fun sum INT_NIL = 0
      | sum (INT_CONS(i, rest)) = i + sum rest 
\end{verbatim}
wzajemnie rekurencyjne:
\begin{verbatim}
    datatype friend_rec = PAT | SUSIE | ANYONE_WHO_GIVES_ME of gift_rec
    and gift_rec = BOTTLES_OF_BEER of int | PHOTO of friend_rec
    fun love PAT = false
      | love SUSIE = true
      | love (ANYONE_WHO_GIVES_ME(a_gift)) = like a_gift
    and like (BOTTLES_OF_BEER(n)) = n > 7
      | like (PHOTO(my_friend)) = love my_friend
\end{verbatim}
polimorficzne :
\begin{verbatim}
    datatype 'a option = SOME of 'a | NONE
    val int_some_list = [SOME(5), NONE, SOME(10), SOME(3), NONE]
    val bool_some_list = [SOME(true), SOME(true), NONE, SOME(true)]
    fun divide (NONE, _) = NONE
      | divide (_, NONE) = NONE
      | divide (SOME(x), SOME(y)) = if y = 0.0 then NONE else SOME(x / y)
\end{verbatim}

\begin{verbatim}
    fun pair_some (SOME(x), SOME(y)) = SOME(x, y)
      | pair_some _ = NONE
    fun divide2 (r, s) = 
        case (pair_some (r, s)) 
          of NONE => NONE
           | (SOME(x, y)) => if y = 0.0 then NONE else SOME(x / y)
\end{verbatim}

Znany nam wbudowany typ \|list| 
jest rekurencyjny i polimorficzny
i ma nast"epuj"ac"a definicj"e:
\begin{verbatim}
    infixr 5 ::
    datatype 'a list = nil | :: of ('a * 'a list)
\end{verbatim}
A drzewo mo"rna zdefiniowa"c na przyk"lad tak:
\begin{verbatim}
    datatype 'a tree = EMPTY | NODE of 'a tree * 'a * 'a tree
\end{verbatim}


\section*{Zadania}
\begin{exercises}

\item
Napisz wielkie mn"ostwo funkcji na drzewach.
Mo"resz napisa"c odpowiedniki wszystkich znanych Ci funkcji na listach.
Mo"resz znale"z"c jakie"s funkcje specyficzne dla drzew, 
na przyk"lad: obchodz"ace drzewo, zamieniaj"ace prawe poddrzewa z lewymi, etc.
Mo"resz napisa"c funkcje z drzew w listy i z list w drzewa. 

\item
Zaimplementuj sortowanie drzewowe (treesort).

Wskaz"owka: W uporz"adkowanym drzewie binarnym liczba wyst"epuj"aca w w"e"zle
jest wi"eksza od wszystkich w lewym poddrzewie i mniejsza od wszyskich w prawym.
Nale"ry zdefiniowa"c dwie g"l"owne funkcje. Jedn"a, kt"ora dostaje list"e liczb naturalnych, 
za"s oddaje uporz"adkowane drzewo binarne zawieraj"ace elementy tej listy. 
Drug"a, kt"ora dostaje uporz"adkowane drzewo binarne, a oddaje
posortowan"a list"e jego element"ow. Treesort to z"lo"renie tych funkcji. 

%\item
%Zaimplementuj sortowanie stogowe (heapsort).

%Wskaz"owka: st"og (heap) to drzewo, w kt"orym ka"rdy element jest mniejszy-r"owny 
%od ka"rdego elementu wyst"epuj"acego pod nim. Nale"ry zdefiniowa"c
%dwie g"l"owne funkcje. Jedn"a, kt"ora dostaje list"e liczb naturalnych, za"s oddaje
%heap zawieraj"acy elementy tej listy. Drug"a, kt"ora dostaje heap, a oddaje
%posortowan"a list"e jego element"ow. Do zrobienia tej drugiej zapewne przyda si"e
%funkcja, kt"ora z dw"och heap"ow robi jeden.

\item
To jest bardzo "ladna (i odrobink"e nieefektywna) definicja liczb naturalnych:
\begin{verbatim} 
    datatype nat = ZERO | SUCC of nat
\end{verbatim}
\|SUCC| od successor - nast"epnik.
Napisz kilka podstawowych funkcji i relacji (czyli funkcji typu \|nat->bool|)
na tych liczbach naturalnych.

\item
Oto drzewo sk"ladni abstrakcyjnej (gramatyka) pewnego j"ezyka wyra"re"n
\begin{verbatim} 
    datatype expression =
        NUM of int
      | TIMES of expression * expression
      | PLUS of expression * expression
\end{verbatim}
Napisz funkcj"e wyliczaj"ac"a warto"s"c wyra"renia.
Napisz wersj"e $LL1$ po\-wy"r\-szej gramatyki.
Napisz funkcj"e przekszta"lcaj"ac"a wyra"renie
w gramatyce $LL1$ w wyra"renie w podanej wy"rej gramatyce.
Napisz parser (u"rywaj"ac gramatyki $LL1$) i lekser.

Wskaz"owka:
\begin{verbatim}
<1st_atom>   -> <numeral>
<2nd_atom>   -> <1st_atom> <2nd_atom'>
<2nd_atom'>  -> epsilon | * <1st_atom> <2nd_atom'>
<3rd_atom>   -> <2nd_atom> <3rd_atom'>
<3rd_atom'>  -> epsilon | + <2nd_atom> <3rd_atom'>
<expression> -> <3rd_atom>
\end{verbatim}

\item
Zdefiniuj typ drzew, w kt"orych ka"rdy wierzcho"lek mo"re mie"c
dowolnie wielu syn"ow. Napisz troch"e funkcji na tych drzewach.

\end{exercises}


%%% Local Variables: 
%%% mode: latex
%%% TeX-master: "TajnySkryptSML"
%%% End: 

\chapter{
Funkcje wy"rszych rz"ed"ow
}

Oto definicja funkcji \|sum|:
\begin{verbatim}
    fun sum [] = 0
      | sum (a::rest) = a + sum rest  
\end{verbatim}
Oto definicja funkcji \|prod|:
\begin{verbatim}
    fun prod [] = 1
      | prod (a::rest) = a * prod rest    
\end{verbatim}
Te funkcje s"a podejrzanie podobne.

R"o"rni"a si"e tylko nazw"a i tym, "re w deficji \|sum| 
w pierwszym r"ownaniu wyst"epuje \|0| a w drugim \|+|,
podczas gdy w definicji \|prod|, w tych samym miejscu 
w pierwszym r"ownaniu jest \|1| a w drugim \|*|.
Chcia"loby si"e napisa"c funkcj"e \|foldr| tak"a, "reby
$\f{sum} = \f{foldr} (+, 0)$ i $\f{prod} = \f{foldr} (*, 1)$.
Nic prostszego:
\begin{verbatim}
    fun foldr (f, x) [] = x 
      | foldr (f, x) (a::rest) = f(a, foldr (f, x) rest)
\end{verbatim}
i tak powsta"la pierwsza funkcja wy"rszego rz"edu \|;-)|.

Teraz mo"rna napisa"c:
\begin{verbatim}
    val sum = foldr (op +, 0)    (* op, because + is an infix identifier *)
    val prod = foldr (op *, 1)
\end{verbatim}
Zauwa"rmy, "re \|foldr| tak jak go zapisali"smy jest polimorficzny:
\begin{verbatim}
    val sum_reals = foldr (op +, 0.0)
    fun and_pair (b, c) = b andalso c
    val and_list = foldr (and_pair, true)
    val list_identity = foldr (op ::, nil)
    val add_sizes = foldr (fn (s, rest_size) => (size s) + rest_size, 0)
\end{verbatim}
(\|fn x => E|, gdzie \|E| to jakie"s wyra"renie (by"c mo"re zawieraj"ace \|x|), 
oznacza funkcj"e kt"ora \|x| przyporz"adkowuje \|E|,
jest to tak zwana anonimowa funkcja zapisana w tak zwanej lambda notacji).

Dla wygody przyj"eto, "re wyra"renie w rodzaju \|(f a) b| (aplikacja funkcji
 \|(f a)| do argumentu \|b|)
mo"rna zapisywa"c bez nawias"ow: \|f a b|.
Przyk"lad: \|(foldr (op +, 0)) [2, 44, 5]|
mo"rna zapisa"c jako: \|foldr (op +, 0) [2, 44, 5]|.

Na zako"nczenie obrazowe podsumowanie dzia"lania \|foldr|:
$$\f{foldr} (f, x) [a_1, a_2, a_3, a_4] = f(a_1, f(a_2, f(a_3, f(a_4, x))))$$
lub w nieco czytelniejszym, infiksowym zapisie:
$$\f{foldr} (+, x) [a_1, a_2, a_3, a_4] = a_1 + (a_2 + (a_3 + (a_4 + x)))$$


\section*{Zadania}
\begin{exercises}

\item%[Zadanie 5.1]
Przy pomocy \|foldr| napisz \|append|.

\item%[Zadanie 5.2]
Przy pomocy \|foldr| napisz \|length|.

\item%[Zadanie 5.3]
Przy pomocy \|foldr| napisz funkcj"e \|good_max|, kt"ora dostawszy
\|n : int| oraz \|l : int list| daje najwi"eksz"a liczb"e
z listy \|l| wi"eksz"a od \|n|,
lub \|n| je"sli w \|l| nie ma "radnej liczby wi"ekszej od \|n|.

\item%[Zadanie 5.4]
Napisz funkcj"e \|map|, kt"ora dostaje funkcj"e \|f| i list"e \|l|,
a daje list"e wynik"ow aplikacji funkcji \|f| do kolejnych element"ow  \|l|.
Pisz"ac prosto:
$$\f{map}\ f\ [a_1, a_2, a_3, a_4] = [f\ a_1, f\ a_2, f\ a_3, f\ a_4].$$
Ten sam \|map| zapisz przy pomocy \|foldr|.

\item%[Zadanie 5.5]
Przy pomocy \|map| napisz funkcj"e \|first_cut|, kt"ora dostaje
list"e list a daje list"e pierwszych element"ow tych list.
Napisz \|first_cut| nie u"rywaj"ac \|map|. 
Czy ta wersja nie przypomina Ci definicji \|map|?

\item%[Zadanie 5.6]
Napisz funkcj"e \|filter|, kt"ora dostaje predykat \|p : 'a -> bool|
i~list"e \|l : 'a list| i~daje list"e sk"ladaj"ac"a si"e z~tych element"ow
\|l|, kt"ore spe"lniaj"a \|p|.

Napisz \|filter| korzystaj"ac z \|foldr|.
Czy ta definicja jest bardziej czytelna?

\item%[Zadanie 5.7]
Przy pomocy \|filter| napisz \|divis_list|, kt"ora wy"lawia
z listy te liczby kt"ore s"a podzielne przez zadan"a liczb"e.
Prawda, "re "ladnie?

\item%[Zadanie 5.8]
Napisz \|foldl|, czyli funkcj"e kt"ora zachowuje si"e w ten spos"ob
$$\f{foldl} (f, x) [a_1, a_2, a_3, a_4] = f(f(f(f(x, a_1), a_2), a_3), a_4)$$
lub w infiksowym zapisie:
$$\f{foldl} (+, x) [a_1, a_2, a_3, a_4] = ((((x + a_1) + a_2) + a_3) + a_4)$$

\item%[Zadanie 5.9]
U"rywaj"ac mi"edzy innymi \|foldl| napisz \|append|.
Zwyk"ly \|append|. 
 
\item%[Zadanie 5.10]
U"rywaj"ac mi"edzy innymi \|foldl| napisz \|foldr|.
Korzystaj z do"swiadcze"n z poprzedniego zadania.

\item%[Zadanie 5.11]
Napisz odpowiedniki \|foldr|, \|map| i \|filter| dla drzew.

\end{exercises}


%%% Local Variables: 
%%% mode: latex
%%% TeX-master: "TajnySkryptSML"
%%% End: 

\chapter{
Funkcje jeszcze wy"rszych rz"ed"ow 
}

Funkcje wy"rszych rz"ed"ow spotyka si"e wsz"edzie. We"zmy cho"cby
aplikacj"e funkcji do argumentu. To"c to tw"or wy"rszego rz"edu! Mo"remy go nazwa"c: 
\begin{verbatim}
    fun apply (f, a) = f a
\end{verbatim}
i od tej pory zawsze pisa"c \|apply (g, b)| tam gdzie dot"ad pisali"smy \|g b  :-]|.

Operacja sk"ladania dw"och funkcji jest oczywi"scie funkcj"a wy"rszego rz"edu. 
Pewna jej wersja jest nazwana i zdefiniowana w standardowej bazie SML-a:
\begin{verbatim}
    infix 3 o
    fun (f o g) x = f (g x)
\end{verbatim}
i czasem si"e przydaje, np.:
\begin{verbatim}
    val treesort = bst2list o list2bst
\end{verbatim}
Nawet pospolite operacje arytmetyczne mo"rna zapisa"c jako funkcje wy"rszego rz"edu:
\begin{verbatim}
    fun divide_by x y = y / x;
    divide_by 2.0 4.0;
    val divide_by_two = divide_by 2.0;
    divide_by_two 10000000.0;
    divide_by_two 342423423452345.0;
    map (divide_by 3.0) [333.0, 12.0, 55.0, 5.0];
\end{verbatim}

Definiuj"ac \|divide_by| na podstawie funkcji ~\|/|~ dokonali"smy pewnych
my\-"slo\-wych operacji wy"rszego rz"edu (jak"reby inaczej). 
Te operacje mo"rna zapisa"c w SML-u:
\begin{verbatim}
    fun curry f a b = f (a, b)
    fun C f x y = f y x
\end{verbatim}
Definicj"e \|divide_by| mo"rna teraz przedstawi"c tak:
\begin{verbatim}
    val divide_by = C (curry (op /))
\end{verbatim}

Tak na marginesie: dla dowolnego \|f : 'a * 'b -> 'c| istnieje dok"ladnie
jedna funkcja \|g : 'a -> 'b -> 'c| taka, "re dla ka"rdego \|(x, y) : 'a * 'b|
\|f (x, y)| $=$ \|g x y|. Tym \|g| jest \|curry f|. Sk"ad wynika na przyk"lad, "re ka"rd"a 
operacj"e arytmetyczn"a mo"rna przerobi"c na funkcj"e wy"rszego rz"edu tylko na dwa sposoby.

Jeszcze jedno --- przy pomocy lambda notacji
funkcje wy"rszych rz"ed"ow zapisuje si"e zagnie"rd"raj"ac \|fn|, np.:
\begin{verbatim}
    fun curry f = fn a => fn b => f (a, b)
    val C = fn f => fn x => fn y => f y x
\end{verbatim}


\section*{Zadania}
\begin{exercises}

\item%[Zadanie 6.1]
Zdefiniuj funkcje odwrotne do \|curry| (b"edzie si"e nazywa"la \|uncurry|)
i \|C| (nazw"e wymy"sl sama (sam)).
Napisz funkcj"e \|repeat|, kt"ora dostawszy pare \|(f, n)| daje
funkcj"e r"own"a z"lo"reniu \|n| funkcji \|f|. Czyli:
$$ \f{repeat}(f,n) = f^n $$

\item%[Zadanie 6.2]
Dla dowolnych dw"och typ"ow \|'a| i \|'b| istnieje typ \|'a * 'b|, ich produkt.
W naturalny spos"ob zwi"azane s"a z tym typem pewne funkcje ---
projekcje:
\begin{verbatim}
    fun pi1 (x, y) = x
    fun pi2 (x, y) = y
\end{verbatim}
i funkcja \|pair| typu \|('c -> 'a) * ('c -> 'b) -> ('c -> 'a * 'b)|,
kt"ora dostawszy funkcje \|f :'c -> 'a| oraz \|g : 'c -> 'b| daje funkcj"e\\
 \|h : 'c -> 'a * 'b|, o tej w"lasno"sci, "re
\|pi1 o h = f| i \|pi2 o h = g|. 
(Taka funkcja \|h| jest dla ka"rdego \|(f, g)| tylko jedna.)

Napisz funkcj"e \|pair|.

\item%[Zadanie 6.3]
Zdefiniuj typ \|('a, 'b) direct_sum|, kt"ory jest sum"a roz"l"aczn"a ty"ow \|'a| i \|'b|.
Ten typ b"edzie w pewnym sensie dualny do typu \|'a * 'b|.
Zdefiniuj w"lo"renia (injekcje):
\begin{verbatim}
    inleft  : 'a -> ('a, 'b) direct_sum 
    inright : 'b -> ('a, 'b) direct_sum
\end{verbatim} 
i funkcj"e \|cases| analogiczn"a (tak naprawd"e dualn"a) do funkcji \|pair| dla produktu.

Wskaz"owka 1:
Funkcja \|cases| jest typu\\
 \|('a -> 'c) * ('b -> 'c) -> ('a, 'b) direct_sum -> 'c|.

Wskaz"owka 2:
Funkcja \|cases| dostawszy funkcje \| f : 'a -> 'c | oraz\\
\| g : 'b -> 'c |
daje funkcj"e \| h : ('a, 'b) direct_sum -> 'c |, o tej w"lasno"sci, "re
\| h o inleft = f | i \| h o inright = g |. 
(Taka funkcja \|h| jest dla ka"rdego \|(f, g)| tylko jedna.)

\pagebreak

\item%[Zadanie 6.4]
Napisz funkcj"e 
\begin{verbatim}
    exists : ('a -> bool) -> 'a list -> bool
\end{verbatim} 
kt"ora bierze predykat
\|p : ('a -> bool)| i daje funkcj"e, kt"ora bierze list"e \|l| i oddaje \|true|
gdy kt"ory"s z element"ow listy spe"lnia predykat \|p|, \|false| w przeciwnym przypadku.
Napisz funkcj"e 
\begin{verbatim}
    non : ('a -> bool) -> ('a -> bool)
\end{verbatim} 
kt"ora neguje predykat.
Napisz funcj"e \|forall|, kt"ora tym si"e r"o"rni od \|exists|, "re w odpowiednim
miejscu nie kt"ory"s, ale wszystkie elementy listy musz"a spe"lnia"c predykat.
Nim zaczniesz pisa"c \|forall| podobnie jak \|exists|, 
pomy"sl czy nie mo"rnaby go napisa"c "smieszniej 
(nie, nie chodzi mi o to "reby u"ryc \|foldr|).

Spoiler:
W definicji \|forall| korzystaj wy"lacznie z \|exists|, \|non|, i \|o|.

\item%[Zadanie 6.5]
%\frenchspacing
Zapisz insertionsort lub inny algorytm sortowania 
(mo"ze mergesort, je"sli jeszcze go nie napisa"le"s lub nie napisa"la"s) w takiej
postaci, aby jako pierwszy argument bra"l funkcj"e \|le : 't * 't -> bool| wyznaczj"ac"a
pewien porz"adek i sortowa"l list"e \|l : 't list| wed"lug tego porz"adku.
Napisz \|le_int| i \|le_string|. 
Wypr"obuj \|insertionsort le_int|.
Napisz te"r jaki"s 
\begin{verbatim}
    le_int_to_int : (int -> int) * (int -> int) -> bool
\end{verbatim}
Wypr"obuj \|insertionsort le_int_to_int|.

\end{exercises}


%%% Local Variables: 
%%% mode: latex
%%% TeX-master: "TajnySkryptSML"
%%% End: 

\chapter{
Operacje na tekstach}

Operacje na tekstach w SML to w~rzeczywisto"sci operacje na listach
znak"ow. Istnieje wbudowana funkcja \|explode|, kt"ora dostaj"ac napis,
daje w wyniku list"e znak"ow, z kt"orych si"e on sk"lada.
Np.\ \|explode "baba"| daje $$\|[#"b",#"a",#"b",#"a"]|$$
lub $$\|["b","a","b","a"]|$$ w zale"rnosci od implementacji.


\section*{Zadania}
\begin{zadania}

\item%[Zadanie 2.6]
U"rywaj"ac funkcji \|explode| zdefiniuj funkcj"e \|size|, 
ktora oblicza d"lugo"s"c napisu.

\item%[Zadanie 2.7]
\|implode| to funkcja odwrotna do \|explode|.
Napisz funkcj"e "l"acz"ac"a dwa napisy.

\item Napisz funkcj"e \|linesplit : string -> string list|,
  przerabiaj"ac"a napis na list"e wierszy, z kt"orych si"e on sk"lada.

\item Napisz funkcj"e \|wordsplit : string -> string list|,
  przerabiaj"ac"a napis na list"e s"l"ow, z kt"orych si"e on sk"lada.

\item Napisz funkcj"e $$\|generic_split : string -> string -> string list|,$$
  przerabiaj"ac"a napis na list"e p"ol, z kt"orych si"e on
  sk"lada, przyjmuj"ac drugi argument funkcji za zbi"or separator"ow p"ol.

\item Napisz funkcj"e sortuj"ac"a leksykograficznie list"e s"l"ow
  (mo"rna, a nawet nale"ry skorzysta"c z napisanej wcze"sniej funkcji
  sortuj"acej)

\item Napisz funkcj"e generuj"ac"a dla danego napisu liste $n$
  najcz"e"sciej w nim wyst"epuj"acych s"l"ow, wraz z~liczb"a
  wyst"apie"n ka"rdego z nich.

\item Napisz funkcj"e  $$\|strchr : string * char -> int list|$$
tak"a, "re \|strchr(s,c)| daje list"e pozycji, na kt"orych znak \|c|
wyst"epuje w napisie \|s|.

\item Napisz funkcj"e  $$\|strstr : string * string -> int list|$$
tak"a, "re \|strstr(s,p)| daje list"e pozycji, na kt"orych zaczynaj"a
si"e wy\-st"a\-pie\-nia napisu  \|p| w napisie \|s|. Uwaga: 
$$\|strstr("ababababa", "babab") = [2,4]|.$$

\end{zadania}

%%% Local Variables: 
%%% mode: latex
%%% TeX-master: "TajnySkryptSML"
%%% End: 

\label{MBchapter}
\chapter{
Leniwo"s"c}
\label{GGchapter}
\chapter{
Modu"ly: struktury i sygnatury
}

Ka"rdy du"ry a dobry program w naturalny spos"ob dzieli si"e 
na szereg odr"ebnych, w sporym stopniu niezale"rnych od siebie podprogram"ow. 
(Cz"esto podzia"l ten nast"epuje ju"r w fazach projektowania i specyfikowania.)
Ka"rdy taki podprogram sk"lada si"e z pewnej liczby typ"ow, 
warto"sci, komentarzy, intuicji, itp.\ i wszystkie jego sk"ladniki s"a mocno ze sob"a logicznie powi"azane.
Je"sli zamanifestujemy istnienie takiego podprogramu deklaruj"ac odpowiedni SML-owy modu"l, 
to zar"owno dla nas, jak i dla kompilatora stanie si"e lepiej widoczne, 
co dok"ladnie wchodzi w sk"lad podprogramu, jakie s"a zale"rnosci mi"edzy jego komponentami, itd.
Je"sli dodatkowo przyjrzymy si"e krytycznie powsta"lemu (lub projektowanemu)
modu"lowi i ocenimy co w nim jest istotne z punktu widzenia reszty naszego programu, 
a co jest jedynie szczeg"o"lem implementacyjnym, 
to b"edziemy mogli opatrzy"c go wzgl"ednie kr"otk"a, a mimo to \emph{dobr"a}, sygnatur"a. 

Przyk"lad, kt"ory wiele wyja"snia:
\begin{verbatim}
signature TOTAL_PREORDER =
sig

    type elem
    val leq : elem * elem -> bool 
    (* and leq should be a total preorder *)

end;

signature PRIORITY_QUEUE =
sig
    
    structure Item : TOTAL_PREORDER

    type queue

    val empty : queue
    val insert : Item.elem * queue -> queue
    val join : queue * queue -> queue 
    datatype min = MIN of Item.elem * queue | EMPTY
    val get_min : queue -> min 

    (* By seq(q) we denote the sequence of Item.elem elements
       produced by iterating get_min on queue q.
       By num(t, q) we denote the number of occurrences of t in seq(q).

       For all q, p - queues constructed using the above operations,
       and for all t, u - elements of type Item.elem:
       1. seq(q) should be finite and non-decreasing,
       2. num(t, empty) should be 0,
       3. num(t, insert(u, q)) should be num(t, q), if u <> t, 
          or num(t, q) + 1, if u = t,
       4. num(t, join(q, p)) should be num(t, q) + num(t, p). *)
    
end;

signature INTEGER_LEQ =
sig

    include TOTAL_PREORDER
    sharing type elem = int
    (* and leq should be the natural less-equal on int *)

end;

structure IntegerLeq : INTEGER_LEQ =
struct

    type elem = int
    fun leq (k, l) = ((k : int) <= l)

end;
    
signature INT_PRIORITY_QUEUE =
sig 

    include PRIORITY_QUEUE
    sharing type Item.elem = int           
    (* and Item.leq should be the natural less-equal on int *)

end;

structure IntPQasList : INT_PRIORITY_QUEUE =
struct

    structure Item = IntegerLeq (* type sharing and requirement *)
                                (* about Item.leq now satisfied *)

    type queue = Item.elem list (* queues will be unordered lists *)
    val empty = nil
    val insert = op ::                            
    val join = op @

    datatype min = MIN of Item.elem * queue | EMPTY
    fun get_min nil = EMPTY
      | get_min (k::tail) = 
        case get_min tail
          of EMPTY => MIN(k, nil)
           | MIN(n, rest) => if Item.leq (k, n) then MIN(k, n::rest)
                             else MIN(n, k::rest)                           

end;
\end{verbatim}

Nazwijmy naszkicowany powy"rej program $IntPQasList$.
Jest on poprawny tzn.\ dobrze implementuje kolejk"e priorytetow"a liczb ca"lkowitych.
W szcze\-g"ol\-no\-"sci \|IntPQasList| pasuje do \|INT_PRIORITY_QUEUE|,
czyli zawiera wszystkie wymagane typy, warto"sci, itp.,
jak r"ownie"r spe"lnia wszystkie wymagania opisane w komentarzach (sprawd"z).

{\bf Definicja} (przez przyk"lad) \emph{dobroci}:
M"owimy, ze sygnatura \|INTEGER_LEQ| jest \emph{dobra} 
dla struktury \|IntegerLeq| w programie $IntPQasList$, gdy"r:
\begin{enumerate}

\item \|IntegerLeq| pasuje do \|INTEGER_LEQ| 

\item jesli napiszemy dowoln"a struktur"e \|A| pasuj"ac"a do \|INTEGER_LEQ|, 
to program $IntPQasList$ z \|IntegerLeq| podmienionym na \|A| pozostanie poprawny

\end{enumerate}

\emph{Dobra} sygnatura pozwala nam i kompilatorowi rozumowa"c o module, jako ca"lo"sci,
wy"l"acznie na podstawie jego sygnatury, bez zagl"adania do kodu.
Wspomaga to abstrakcyjne my"slenie o programie, 
bardzo u"latwia piel"egnacj"e i modyfikacj"e, 
pozwala na roz"l"aczna kompilacj"e modu"l"ow, etc.

Gdyby w programie $IntPQasList$ zamiast \|INTEGER_LEQ| 
zdefiniowa"c nast"epuj"ac"a sygnatur"e:
\begin{verbatim}
signature INTEGER_LEQ' =
sig

    include TOTAL_PREORDER
    sharing type elem = int
    (* and leq should additionally be antisymmetric *)

end;
\end{verbatim}

to nie by"laby ona \emph{dobr"a} sygnatur"a dla \|IntegerLeq|.
Oto dow"od:

\begin{verbatim}
structure A : INTEGER_LEQ' =
struct

    type elem = int
    fun leq (k, l) = ((k : int) >= l) (* comment satisfied *)

end;
\end{verbatim}

Wida"c, "re \|A| pasuje do \|INTEGER_LEQ'|. 
Ale program $IntPQasList$ z \|INTEGER_LEQ'| zamiast \|INTEGER_LEQ|,
gdy podmienimy w nim \|IntegerLeq| na \|A|, przestaje by"c poprawny, 
gdy"r porz"adek na \|Item.elem| w module \|IntPQasList| 
jest odwrotny ni"r wymagany w jego specyfikacji.

\bigskip
{\bf Pisz zawsze \emph{dobre} sygnatury!}


\section*{Zadania}
\begin{zadania}         % Ten sam efekt daje \begin{exercises}, he he.

\item
Napisz modu"ly \|COMPLEX_LEQ| (liczby zespolone uporz"adkowane wed"lug ich modu"l"ow), 
\|COMPLEX_PRIORITY_QUEUE|, \|ComplexLeq|, \|ComplexPQasList|. Sprawd"z poprawno"s"c i przetestuj.

\item
Napisz program $IntPQasOrderedList$. Upewnij si"e, "re modu"l \|IntPQasOrderedList| pasuje do sygnatury \|INT_PRIORITY_QUEUE|.
Pomy"sl o z"lo"rono"sci czasowej $IntPQasList$ oraz $IntPQasOrderedList$.

\item
Napisz program sortuj"acy liczby ca"lkowite przy u"ryciu kolejki priorytetowej. 
Wykorzystaj $IntPQasList$ jako jego podprogram. 
Czy \|INT_PRIORITY_QUEUE| jest \emph{dobr"a} sygnatur"a 
dla \|IntPQasList| w tym programie?

\item
Zmodyfikuj program z poprzedniego zadania tak,
aby korzysta"l z $IntPQasOrderedList$.
Jaki algorytm sortowania wynika z u"rycia \|insert| 
a jaki z u"rycia \|join| przy tworzeniu kolejki?

\end{zadania}


%%% Local Variables: 
%%% mode: latex
%%% TeX-master: "TajnySkryptSML"
%%% End: 

\chapter{
Modu"ly: funktory
}

Przy pomocy funktora mo"rna wyra"znie opisa"c, jakie modu"ly
s"a potrzebne by zaimplementowa"c dany modu"l. Gdy program jest napisany 
w ca"lo"sci przy pomocy funktor"ow, w naturalny spos"ob przybiera on hierarchiczn"a posta"c.
Jest jeden modu"l, kt"ory rz"adzi i kt"orego sygnatura jest opisem naszych
oczekiwa"n co do programu, s"a modu"ly spe"lniaj"ace specyfikacje argument"ow
tego rz"adz"acego funktora, s"a modu"ly spe"lniaj"ace specyfikacje argument"ow tych modu"l"ow, itd.
Dodatkowo, je"sli funktor ma dobr"a sygnatur"e i nie wymienia si"e w nim nazw "radnych zewn"etrznych
modu"l"ow (element dobrego stylu), to mo"rna o nim rozumowa"c lub zmienia"c
jego implementacj"e, abstrahuj"ac od w"lasno"sci czy istnienia jakichkolwiek innych modu"l"ow
i nie psuje to poprawno"sci programu.  

Przyk"lad funktora:
\begin{verbatim}
... (* <- TREE and Tree are defined here *)   

functor PQasHeap 
    ( structure TotalPreorder : TOTAL_PREORDER
      structure Tree : TREE ) :
sig 

    include PRIORITY_QUEUE
    sharing Item = TotalPreorder

    (* queues should be constructed as trees  *)
    (* with value of every node less or equal *)
    (* to values of all the descendant nodes  *)

end =
struct

    structure Item = TotalPreorder (* sharing satisfied *)

    type queue = Item.elem Tree.tree 
    ... (* still some work ;-) *)

end;

structure IntPQasHeap = PQasHeap (structure TotalPreorder = IntegerLeq 
                                  structure Tree = Tree);
\end{verbatim}

\section*{Zadania}
\begin{zadania} 

\item
Napisz program $PQasHeap$ wed"lug szkicu podanego powy"rej. 
Sprawd"z, czy sygnatury s"a dobre.
Przetestuj w przypadku liczb ca"l\-ko\-wi\-tych i zespolonych.

\item
Zmie"n \|TREE| i \|Tree| tak, by \|Tree| by"l funktorem, o argumencie \|Item|,
a \|tree| nie by"lo polimorficznym typem danych, tylko typem zale"rnym od argumentu funktora:
\begin{verbatim}
datatype tree = EMPTY | NODE of tree * Item.item * tree.  
\end{verbatim}

U"ryj tego \|TREE| i \|Tree| do zrobienia innej wersji programu $PQasHeap$.
To jest niezbyt "ladne bez funktor"ow wy"rszego rz"edu, ale wykonalne.

\item
Zmodyfikuj $PQasHeap$ tak, by drzewa, jakimi s"a kolejki, 
by"ly zawsze utrzymywane w postaci wywa"ronej.
Postaraj si"e w jak najwi"ekszym fragmencie programu
abstrahowa"c od sposobu w jaki zapewniane jest wywa"renie.
Zamanifestuj t"e abstrakcj"e odpowiednio deklaruj"ac modu"ly.

\item
Napisz funktor implementuj"acy sortowanie przy pomocy kolejek priorytetowych.
Pomy"sl o z"lo"rono"sci czasowej sortowania przy u"ryciu $PQasHeap$.

\end{zadania}

\bigskip

{\bf To ju"r koniec :-[}


%%% Local Variables: 
%%% mode: latex
%%% TeX-master: "TajnySkryptSML"
%%% End: 

\appendix
\chapter{
Jak zaliczy"c laboratorium}

Aby zaliczy"c laboratorium SML nale"ry napisa"c program.
Powinien si"e on sk"la\-da"c z kilku do kilkunastu ma"lych
modu"l"ow; dobrze wyspecyfikowanych, po\-"l"a\-czo\-nych
w sensown"a ca"lo"s"c jasnymi, wyra"znie opisanymi zale"rno"sciami.

Temat jest dowolny. S"lu"rymy rad"a i pomoc"a przy jego wyborze.
Oto przyk"ladowe (troch"e nudne i wykorzystywane ju"r zbyt wiele razy) zadanie zaliczeniowe:

\bigskip

\bigskip
Napisa"c $Kalkulator$, czyli program, kt"ory w"sr"od swoich modu"l"ow ma modu"l
pasuj"acy do sygnatury \|CALCULATOR|, zdefiniowanej nast"epuj"aco:

\begin{verbatim}
signature CALCULATOR =
sig
    
    val eval : string -> string

    (* if s represents an arithmetic expression    *)
    (* (which means that s is a string of digits,  *)
    (* ~ (negation), +, -, * and parenthesis,      *)
    (* satisfying some well known properties),     *)
    (* then (eval s) should be equal to the string *) 
    (* representing the value of s, or the string  *)
    (* "division by zero" if appropriate,          *)
    (* if s does not represent a valid arithmetic  *)
    (* expression, (eval s) should be equal to the *)
    (* string "syntax error"                       *)

end;
\end{verbatim}

Wszelkie drobne niejasno"sci tej specyfikacji nale"ry interpretowa"c 
na swoj"a korzy"s"c. W razie problem"ow zwr"oci"c si"e o pomoc.

Wydaje si"e, "re w"sr"od modu"l"ow sk"ladaj"acych si"e na $Kalkulator$ 
powinny by"c modu"ly \|Lexer|, \|Parser|, \|Evaluator| i \|PrettyPrinter|. 
By"loby "ladnie i chyba wygodnie, gdyby \|Parser|
pos"lugiwa"l si"e drzewem sk"ladni reprezentuj"acym gramatyk"e $LL(1)$ wyra"re"n,
za"s \|Evaluator| drzewem reprezentuj"acym prostsz"a gramatyk"e:
\begin{verbatim} 
    datatype expression =
        NUM of int
      | NEGATION of expression
      | TIMES of expression * expression
      | PLUS of expression * expression
      | ...
\end{verbatim}

To zadanie nie powninno by"c zbyt trudne.
To co mo"rna tu zrobi"c lepiej lub gorzej to podzia"l programu na modu"ly,
zdefiniowanie zale"rno"sci mi"edzy modu"lami przy pomocy funktor"ow,
opisanie w"lasno"sci modu"l"ow sygnaturami.


\vspace{0.3cm}

{\bf Powodzenia!}

%%% Local Variables: 
%%% mode: latex
%%% TeX-master: "TajnySkryptSML"
%%% End: 



\chapter{
Rozwi"azania niekt"orych zada"n
}

Wszystkie te rzeczy przesz"ly przez kompilator
(w wypadku program"ow w SML to bardzo dobry omen),
ale nie by"ly zbyt intensywnie testowane. 
Dzi"eki za wskazywanie b"l"ed"ow!                     

\begin{verbatim}

(* Exercise 2.1 *) 

exception Empty
fun tl [] = raise Empty
  | tl (head::tail) = tail

(* Exercise 2.2 *) 

fun sum l = if null l then 0 else hd l + sum (tl l)

(* or *)

fun sum [] = 0
  | sum (i::rest) = i + sum rest  

(* Exercise 2.3 *)

fun length l = if null l then 0 else 1 + length (tl l)

(* or *)

fun length [] = 0
  | length (head::tail) = 1 + length tail  

(* Exercise 2.4 *)

fun last l = if null l then hd l else last (tl l)

(* or *)

fun last [] = raise Empty
  | last [elem] = elem
  | last (head::tail) = last tail

(* Exercise 2.5 *)

fun append ([], l) = l
  | append (head::tail, l) = head :: (append (tail, l)) 

(* Exercise 2.6 *)

fun take (l, 0) = nil
  | take (head::tail, n) = head :: take (tail, n-1)
  | take (nil, n) = nil

fun drop (l, 0) = l
  | drop (head::tail, n) = drop (tail, n-1)
  | drop (nil, n) = nil

(* Check, that the required equality          *)
(* holds even if n < 0. This is an ugly hack, *)
(* I know, but I hate exceptions even more:P. *)

(* Exercise 2.7 *)

fun nth (l, n) = hd (drop (l, n-1))

(* And this version is slower. Why? *)

fun nth2 (l, n) = last (take (l, n))

(* Exercise 2.8 *)

fun flatten [] = []
  | flatten (l::rest) = append (l, flatten rest)

(* Exercise 2.9 *)

(* @ is the built in, infix version of append *)     

fun quicksort [] = [] : int list
  | quicksort [x] = [x]
  | quicksort (a::rest) = 
    let 
        fun split (b, []) = ([], [])
          | split (b, x::l) = let 
                                  val (left, right) = split (b, l)
                              in
                                  if x < b then (x::left, right)
                                  else (left,x::right)
                              end
    in 
        let
            val (left, right) = split (a, rest)
        in
            quicksort left @ (a::quicksort right)
        end
    end

(* Exercise 3.1 *) 

fun length l = let
                   fun len_acc (len, []) = len
                     | len_acc (len, head::tail) = len_acc (len + 1, tail)
               in
                   len_acc (0, l)
               end

(* Exercise 3.2 *)

fun decimal l = 
  let fun diggy (acc, []) = acc
        | diggy (acc, digit::rest) = diggy (10*acc + digit, rest)
  in
      diggy(0, l)
  end

(* Definitions of decimal without an accumulator are absolutely hopeless *)

(* Exercise 3.3 *)

(* reverse2 is O(n^2) *)

fun reverse2 [] = []
  | reverse2 (head::tail) = (reverse2 tail) @ [head] 

(* Thanks to the use of accumulator reverse is O(n) *)
(* and by chance there is tail recursion here *)

fun reverse l = let
                    fun rev_acc (rev, []) = rev
                      | rev_acc (rev, head::tail) = rev_acc (head::rev, tail) 
                in
                    rev_acc ([], l)
                end



(* Exercise 3.4 *)

(* This works. Notice reverse.               *)
(* A similar thing could be done for append. *)
(* Not worth the effort, is it?              *)

exception Empty

fun take (l, n) = 
    let
        fun take_acc (acc, (l, 0)) = reverse acc
          | take_acc (acc, (hd::tl, n)) = take_acc (hd :: acc, (tl, n-1))
          | take_acc (acc, (nil, n)) = raise Empty
    in
        take_acc ([], (l, n))
    end

(* Exercise 3.5 *)

fun interval (from, to) = 
    let
        fun inter_acc (l, (n, m)) = if n > m then l 
                                    else inter_acc (m::l, (n, m-1))
    in
        inter_acc ([], (from, to))
    end

(* Exercise 3.6 *)

val minus_infinity = ~10000000000000000000000000000000.0

fun real_max [] = minus_infinity
  | real_max (head::tail) = 
    let fun maxx (r, []) = r
          | maxx (r, x::tl) = if r > x then maxx (r, tl) 
                              else maxx (x, tl)
    in
        maxx (head, tail)
    end

fun real_max [] = minus_infinity
  | real_max (x::tail) = let 
                        val y = real_max tail
                    in
                        if y > x then y
                        else x
                    end



(* Exercise 5.1 *) 

fun foldr (f, x) [] = x 
  | foldr (f, x) (a::rest) = f(a, foldr (f, x) rest)

fun append (l1, l2) = foldr (op ::, l2) l1

(* Exercise 5.2 *) 

val length = foldr (fn (elem, rest_length) => 1 + rest_length, 0) 

(* Exercise 5.3 *) 

fun good_max n = foldr (max, n)

(* Exercise 5.4 *) 

fun map f nil = nil 
  | map f (head::tail) = (f head)::(map f tail)

fun map f = foldr (fn (a, result) => (f a):: result, nil)

(* Exercise 5.5 *) 

val first_cut = map hd

fun first_cut nil = nil
  | first_cut (head::tail) = (hd head)::(first_cut tail)

(* Exercise 5.6 *) 

fun filter p nil = nil
  | filter p (a::rest) = if p a then a::(filter p rest)
                         else (filter p rest)

fun filter p = foldr (fn (a, filtered_tail) => 
                      if p a then a::filtered_tail
                      else filtered_tail, nil)                       
(* No, this didn't turn out more readable (at least for me). *)

(* Exercise 5.7 *) 

fun divisible_by k l = l mod k = 0
fun divis_list n = filter (divisible_by n)
(* Yeees. *)

(* Exercise 5.8 *) 

fun foldl (f, x) nil = x
  | foldl (f, x) (head::tail) = foldl (f, (f (x, head))) tail

(* Exercise 5.9 *) 

fun reversed_cons (tail, head) = head :: tail 
fun append (l1, l2) = foldl (reversed_cons, l2) (rev l1)

(* Exercise 6.1 *) 

fun uncurry f = fn (a, b) => f a b

fun id x = x
fun repeat (f, 0) = id
  | repeat (f, n) = f o (repeat (f, n - 1))

(* Exercise 6.2 *) 

fun pair (f, g) x = (f x, g x)

(* Exercise 6.3 *) 
 
datatype ('a, 'b) direct_sum = INL of 'a | INR of 'b
val inleft = INL
val inright = INR
fun cases (f, g) (INL(x)) = f x
  | cases (f, g) (INR(y)) = g y

(* Exercise 6.4 *) 

fun exists p [] = false 
  | exists p (a::rest) = if p a then true 
                         else exists p rest
fun non p = not o p
val forall = non o exists o non
\end{verbatim}

%exception 
%WeDoNotKnowYetHowToDefineNaturalNumbersSoWeUseIntegersAnd_COMPLAIN
%
%fun take (l, n) = 
%    if n < 0 then raise 
%WeDoNotKnowYetHowToDefineNaturalNumbersSoWeUseIntegersAnd_COMPLAIN
%    else let
%             fun take' (l, 0) = nil
%               | take' (head::tail, n) = head :: take (tail, n-1)
%               | take' (nil, n) = raise Empty
%         in
%             take' (l, n)
%         end

%%% Local Variables: 
%%% mode: latex
%%% TeX-master: "TajnySkryptSML"
%%% End: 

\chapter{
Jak pracowa"c z kompilatorem SML-a}

\begin{itemize}

\item Interaktywnie:
\begin{enumerate}
\item uruchomi"c SML (\|mosml|, \|new-sml|, \|sml-0.93|, \|sml|)
\item po zach"ecie, kt"or"a na og"o"l jest my"slnik, napisa"c program
\item zako"nczy"c pisanie "srednikiem i nacisn"a"c \|RET|
\end{enumerate}

\item Wsadowo: 

\begin{enumerate}
\item napisa"c program jakim"s edytorem
\item zapisa"c go w pliku \|smth.sml|
\item uruchomi"c SML i napisac \|use "smth.sml";|
\end{enumerate}

\item Przy pomocy EMACS-a i \|sml-mode|:

\begin{enumerate}
\item uruchomi"c EMACS-a
\item \|C-x C-f smth.sml RET| (wa"rne, "reby ko"nc"owk"a nazwy by"lo \|.sml|)
\item napisa"c b"l"edny program (\|Tab| automagicznie formatuje)
\item \|C-c C-l| i b"edzie wiadomo co dalej
\item Zrobi"c \|C-c `| (ten ostatni znaczek to back-quote, 
      wyst"epuje na klawiaturze na og"o"l pod tyld"a), 
      wtedy on ustawi kursor tam gdzie w pliku jest b"l"ad 
\item Korzysta"c z~innych dobrodziejstw \|sml-mode|,
      np.\ \|sml-mode-info|, czyli podr"ecznika do \|sml-mode|
\end{enumerate}

\end{itemize}

%%% Local Variables: 
%%% mode: latex
%%% TeX-master: "TajnySkryptSML"
%%% End: 


\end{document}


%%% Local Variables: 
%%% mode: latex
%%% TeX-master: t
%%% End: 
