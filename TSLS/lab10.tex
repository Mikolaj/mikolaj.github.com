\chapter{
Modu"ly: funktory
}

Przy pomocy funktora mo"rna wyra"znie opisa"c, jakie modu"ly
s"a potrzebne by zaimplementowa"c dany modu"l. Gdy program jest napisany 
w ca"lo"sci przy pomocy funktor"ow, w naturalny spos"ob przybiera on hierarchiczn"a posta"c.
Jest jeden modu"l, kt"ory rz"adzi i kt"orego sygnatura jest opisem naszych
oczekiwa"n co do programu, s"a modu"ly spe"lniaj"ace specyfikacje argument"ow
tego rz"adz"acego funktora, s"a modu"ly spe"lniaj"ace specyfikacje argument"ow tych modu"l"ow, itd.
Dodatkowo, je"sli funktor ma dobr"a sygnatur"e i nie wymienia si"e w nim nazw "radnych zewn"etrznych
modu"l"ow (element dobrego stylu), to mo"rna o nim rozumowa"c lub zmienia"c
jego implementacj"e, abstrahuj"ac od w"lasno"sci czy istnienia jakichkolwiek innych modu"l"ow
i nie psuje to poprawno"sci programu.  

Przyk"lad funktora:
\begin{verbatim}
... (* <- TREE and Tree are defined here *)   

functor PQasHeap 
    ( structure TotalPreorder : TOTAL_PREORDER
      structure Tree : TREE ) :
sig 

    include PRIORITY_QUEUE
    sharing Item = TotalPreorder

    (* queues should be constructed as trees  *)
    (* with value of every node less or equal *)
    (* to values of all the descendant nodes  *)

end =
struct

    structure Item = TotalPreorder (* sharing satisfied *)

    type queue = Item.elem Tree.tree 
    ... (* still some work ;-) *)

end;

structure IntPQasHeap = PQasHeap (structure TotalPreorder = IntegerLeq 
                                  structure Tree = Tree);
\end{verbatim}

\section*{Zadania}
\begin{zadania} 

\item
Napisz program $PQasHeap$ wed"lug szkicu podanego powy"rej. 
Sprawd"z, czy sygnatury s"a dobre.
Przetestuj w przypadku liczb ca"l\-ko\-wi\-tych i zespolonych.

\item
Zmie"n \|TREE| i \|Tree| tak, by \|Tree| by"l funktorem, o argumencie \|Item|,
a \|tree| nie by"lo polimorficznym typem danych, tylko typem zale"rnym od argumentu funktora:
\begin{verbatim}
datatype tree = EMPTY | NODE of tree * Item.item * tree.  
\end{verbatim}

U"ryj tego \|TREE| i \|Tree| do zrobienia innej wersji programu $PQasHeap$.
To jest niezbyt "ladne bez funktor"ow wy"rszego rz"edu, ale wykonalne.

\item
Zmodyfikuj $PQasHeap$ tak, by drzewa, jakimi s"a kolejki, 
by"ly zawsze utrzymywane w postaci wywa"ronej.
Postaraj si"e w jak najwi"ekszym fragmencie programu
abstrahowa"c od sposobu w jaki zapewniane jest wywa"renie.
Zamanifestuj t"e abstrakcj"e odpowiednio deklaruj"ac modu"ly.

\item
Napisz funktor implementuj"acy sortowanie przy pomocy kolejek priorytetowych.
Pomy"sl o z"lo"rono"sci czasowej sortowania przy u"ryciu $PQasHeap$.

\end{zadania}

\bigskip

{\bf To ju"r koniec :-[}


%%% Local Variables: 
%%% mode: latex
%%% TeX-master: "TajnySkryptSML"
%%% End: 
